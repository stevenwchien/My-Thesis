Threshold secret sharing schemes are procedures in which groups of a sufficient size can work together to recover a shared secret. In this thesis, we analyze quantum threshold schemes, which are threshold secret sharing schemes applied to quantum information. Many of the restrictions on quantum secret sharing schemes arise from the no-cloning theorem. We investigate the potential benefit of implementing quantum threshold schemes using two or more identical copies of a secret quantum state. This idea is motivated by the possibility that the availability of multiple copies of the secret can circumvent the restrictions imposed by the no-cloning theorem. Our approach takes advantage of the multiple copies by using the union of two or more access structures, one for each quantum state, in order to implement secret sharing schemes that would otherwise not be realizable. We find that we are indeed able to implement a wider range of access structures, but we show that we can only realize one new threshold scheme for every new copy of the share, given a fixed number of players.
