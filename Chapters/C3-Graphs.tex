\chapter{Graphs}
\label{ch:graphs}

Access structures lend themselves naturally to graphical representations. Here, we show a few basic definitions and results from graph theory, as they will become useful in our discussion.

\begin{definition}{Graph.}
    \label{defn:graph}
    A \textbf{graph} $G=(V,E)$ is composed of a set of vertices and a set of edges. Each edge is incident to two vertices.
\end{definition}

We say that two vertices are \textbf{adjacent} if they are incident to the same edge.

\begin{definition}{Vertex Induced Subgraph.}
    \label{defn:vert-induced-subgraph}
    A \textbf{vertex-induced subgraph} of $G=(V,E)$ by the vertex set $V'$ is the graph $G'$ with vertex set $V'$ and edge set consisting those edges with both endpoints in $V'$. This is also called an \textit{induced subgraph}.
\end{definition}

\begin{definition}{Complement.}
    \label{defn:graph-complement}
    The \textbf{complement} of a graph $G=(V,E)$ is the graph $G^c = (V,E^c)$. There is an edge between two vertices $u,v \in V$ in $G^c$ if $u$ and $v$ are not adjacent in $G$. 
\end{definition}

\begin{definition}{Stable Set.}
    \label{defn:stable-set}
    A \textbf{stable set} in a graph $G=(V,E)$ is a subset of vertices $V_s \subseteq  V$ where there are no edges $e \in E$ that have both endpoints in $V_s$.
\end{definition}

\begin{definition}{Bipartite Graph.}
    \label{defn:bipartite}
	A \textbf{bipartite graph} is a graph in which the vertices can be separated into two stable sets.
\end{definition}

Bipartite graphs are an important class of graphs because of their wide applicability, and because of their relatively simple structure, there exist numerous results regarding them. One elementary result provides both a necessary and sufficient condition for bipartite graphs.

\begin{nonumtheorem}
    \label{thm:bipartite-odd}
    A graph is bipartite if and only if it does not contain any odd cycles.
\end{nonumtheorem}

\begin{definition}{Complete Graph.}
    \label{defn:clique}
    A \textbf{complete graph} is a graph where each vertex is adjacent to every other vertex. A complete graph with $n$ nodes is often denoted as $K_n$.
\end{definition}

If a vertex-induced subgraph is a complete graph, then we call that induced subgraph a \textbf{clique}.

\begin{remark}
    Any induced subgraph of a complete graph is a clique.
\end{remark}

\begin{definition}{Chromatic Number.}
    \label{defn:colors}
	A vertex coloring of a graph $G$ assigns each vertex in $V$ a color such that two vertices of the same color are not adjacent. The \textbf{chromatic number} of a graph $\chi(G)$ is the minimum number of colors needed to do a vertex coloring on $G$.
\end{definition}

\begin{remark}
    The chromatic number of $K_n$ is $n$. The chromatic number of a bipartite graph is $2$.
\end{remark}

\begin{definition}{Clique Cover Number.}
    \label{defn:clique-cover-num}
    The \textbf{clique cover number} is the minimum number of cliques needed to cover the vertex set $V$ of $G$.
\end{definition}

\begin{proposition}
    \label{prop:chrom-clique}
    The clique cover number of a graph $G$ is equal to the chromatic number of the complement of the graph $G^c$:
    \begin{align}
        \theta(G) = \chi(G^c)
    \end{align}
\end{proposition}

\begin{proof}
    Each clique in $G$ is a stable set in $G^c$. In a vertex coloring of $G$, vertices that form a stable set can all be the same color. Therefore if you find a partition of the vertices in a graph such that each subset of vertices is a stable set, the minimum cardinality of that partition is the same as its chromatic number.
\end{proof}

The following definition will be useful in our analysis in later chapters.

\begin{definition}{Kneser Graph.}
    \label{defn:kneser-graph}
    A \textbf{Kneser Graph}, denoted $K(a,b)$, is a graph with the set of all $b$-subsets of $a$ as the vertex set. There is an edge between two vertices if the subsets that they represent are disjoint. This graph has $\binom{a}{b}$ vertices.  
\end{definition}

This class of graphs was introduced by Lova\'{s}z in 1978 \cite{lovasz_knesers_1978}, which he used to prove Kneser's Conjecture, originally posed by Martin Kneser in 1955.

\begin{kneserconjecture}
    \label{thm:kneser-conjecture}
    Whenever the $t$-subsets of a $(2t+j)$-set are divided into $j+1$ classes, then two disjoint subsets end up in the same class. 
\end{kneserconjecture}

The above conjecture, which has now been proven, is stated as a set theoretical result, and is equivalent to the following result on Kneser graphs:

\begin{corollary}
    \label{cor:no}
    Let $k=j+1$. $K(2t-1+k,t)$ \textbf{is not} $k$-colorable.
\end{corollary}

% This will be the shortest proof ever in the history of proofs.
\begin{proof}
     This result follows directly from the definition of a Kneser Graph. A $K(2t-1+k,t)$ has a vertex for every $t$-subset of $2t-1+k$. There is an edge between two vertices if they are disjoint. Consider a minimum vertex coloring on this graph. If two vertices represent disjoint sets, then they must be adjacent. So in the vertex coloring, they cannot be the same color. \hyperref[thm:kneser-conjecture]{Kneser's Conjecture}  states that if we attempt to split the vertices of the graph into $k$ separate groups, then there must be at least one pair of adjacent vertices that end up in the same group. Therefore, a $k$-coloring cannot exist for this graph, so $K(2t-1+k,t)$ is not $k$-colorable.
\end{proof}