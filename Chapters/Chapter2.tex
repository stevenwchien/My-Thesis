\chapter{Background}

\section{Classical Secret Sharing}

In this section we will engage in a rigorous discussion about classical secret sharing and some of the more common schemes. Let us say that the secret/data that we would like to share is called $\mathcal{S}$. In most schemes, there is a trusted dealer $\mathcal{D}$ that has access to the entire secret. The dealer $\mathcal{D}$ distributes shares of $\mathcal{S}$ among a set of participants $\mathcal{P}$. Below, we will provide some definitions that will aid our discussion.

\theoremstyle{definition}
\begin{definition}{Perfect Secret Sharing Scheme.}
    A secret sharing scheme among $n$ individuals is perfect if only the authorized subsets of those individuals can access the secret, and unauthorized subsets have no additional information about the secret.
\end{definition}

One of the most common and most applicable secret sharing schemes is the threshold scheme.

\theoremstyle{definition}
\begin{definition}{Threshold Scheme.}
    A $(t,n)$ threshold scheme is a secret sharing scheme among $n$ individuals such that at least $t$ of those individuals must work together to access the secret. Any group of smaller size should have no additional information about the secret if the scheme is perfect, and at most incomplete information otherwise. 
\end{definition}

% finish this one
 The threshold scheme for classical information has a pretty simple implementation. In 1999 Shamir developed an implementation for a perfect threshold scheme based on polynomial interpolation...
 
 \section{General Access Structures}
 
 The threshold scheme is only a specific example of a secret sharing scheme. What we 

\theoremstyle{definition}
\begin{definition}{Access Structure.}
    An access structure is often denoted as $\Gamma$. An access structure specifies the set of all authorized subsets of users that are able to recover the secret.
\end{definition}

 We say that $A$ is authorized if $A \in \Gamma$. So, for our $(t,n)$ threshold scheme defined above, the access structure can be defined more formally as such: $\Gamma = \{A | A \subseteq \mathcal{P} , |A| \geq t\}$.
 
 An example of an access structure for $\mathcal{P} = \{p_1,p_2,p_3\}$ might be $\Gamma = \{p_1p_2,p_2p_3,p_3p_1\}$. This access structure describes a $(2,3)$ threshold scheme. To be specific this is a minimal access structure, consisting of all minimal authorized subsets.
 
\section{Hypergraph Formulation}

Access structures lend themselves naturally to graphical representations. One way of representing access structures is using a \textit{hypergraph}.

\theoremstyle{definition}
\begin{definition}{Hypergraph.}
    A hypergraph $H$ is a pair $(V,E)$ that specify a set of vertices $V$ and a set of hyperedges $E$. A hyperedge is the extension of a normal edge where the edge can have any number of non-zero endpoints, up to the number of vertices in the graph.
\end{definition}

\theoremstyle{definition}
\begin{definition}{Hypercycles.}
    A hypercycle $H$ is a hypergraph $H = (V,E)$ where there exists an ordering of hyperedges: $(E_1, E_2, \cdots, E_{m-1})$ such that $E_{i} \cap E_{(i+1) m o d m} \neq \emptyset$ for $i = 1,2,\cdots,m-1$.
\end{definition}

\theoremstyle{definition}
\begin{definition}{Hyperstars.}
    A hyperstar is a hypergraph $H = (V,E)$ where the intersection of all the hyperedges $\bigcap_{E_{i} \in E} E_{i} \neq \varnothing$.
\end{definition}



\section{Quantum Computing Preliminaries}

- Quantum State

- Quantum Operation

- No cloning

\section{Quantum Secret Sharing}

\theoremstyle{definition}
\begin{definition}{Quantum Secret Sharing.}
    A quantum secret sharing scheme (QTS) is a threshold secret sharing scheme applied to a quantum secret. This secret is denoted as $\ket{\Psi}$. As in the classical case, we take a quantum secret and divide it into shares to distribute among a set of individuals. We will denote a QTS among $n$ individuals with a threshold of $t$ as $[t,n]$, using square brackets to denote that the scheme is quantum.
\end{definition}

\theoremstyle{definition}
\begin{definition}{Quantum Threshold Scheme.}
    A quantum threshold scheme (QTS) is a threshold secret sharing scheme applied to a quantum secret. This secret is denoted as $\ket{\Psi}$. As in the classical case, we take a quantum secret and divide it into shares to distribute among a set of individuals. We will denote a QTS among $n$ individuals with a threshold of $t$ as $[t,n]$, using square brackets to denote that the scheme is quantum.
\end{definition}

\theoremstyle{theorem}
\begin{theorem}
\label{qss2}
    A QSS scheme exists for a general access structure $\Gamma$ only if for every $A_1, A_2 \in \Gamma$, $A_1 \cap A_2 \neq \emptyset$.
\end{theorem}

\begin{proof}
    Let's assume, for the sake of contradiction, that there exist two disjoint subsets such that each recover the secret. Then, we would have two copies of the quantum state, which violates the no-cloning theorem. 
\end{proof}

\theoremstyle{theorem}
\begin{theorem}
    A QTS $[t,n]$ exists iff $t > \frac{n}{2}$.
\end{theorem}

\begin{proof}
    If we have $t \leq \frac{n}{2}$, then there exist two disjoint subsets such that can each recover the secret. This 
\end{proof}

Let us now discuss the implementation of a QTS.

\theoremstyle{theorem}
\begin{theorem}
    An access structure $\Gamma$ is possible for a QSS scheme iff the access structure is monotone.
\end{theorem}

\theoremstyle{theorem}
\begin{theorem}
    An access structure $\Gamma$ is possible for a QSS scheme iff the intersection of any two authorized subsets is nonempty.
\end{theorem}

As we can see, the no-cloning theorem imposes most of the limitations on realizable QSS schemes. In the next section, we will explore possibilities that we have to potentially \textbf{circumvent} these limitations by relaxing our constraints. 

