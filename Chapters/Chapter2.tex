\chapter{Background}

\section{Classical Secret Sharing}

In this section we will describe secret sharing schemes and talk about some of its properties and definitions. A secret sharing scheme allows some \textit{secret} $\mathcal{S}$ to be divided into shares and distributed by some \textit{dealer} $\mathcal{D}$ to a set of \textit{participants} or \textit{players} $\mathcal{P}$. Each secret sharing scheme has a corresponding \textit{access structure} denoted by $\Gamma$. $\Gamma$ defines the set of \textit{authorized subsets} of $\mathcal{P}$ that have access to the secret. A subset of participants $A \in \Gamma$ should be able to reconstruct the original secret in its entirety, but a a set $B \notin \Gamma$ should have \textbf{no} information about the secret, in the sense that all possible values of the secret are equally likely. Let us present some more formalized definitions:

\theoremstyle{definition}
\begin{definition}{Access Structure.}
    \label{defn:access-structure}
    An \textbf{access structure} is often denoted as $\Gamma$. An access structure specifies the set of all authorized subsets of players that are able to recover the secret.
\end{definition}

\theoremstyle{definition}
\begin{definition}{Authorized Set.}
    \label{defn:authorized-set}
    An \textbf{authorized set} is a subset $A \subseteq \mathcal{P}$ such that $A \in \Gamma$ for an access structure $\Gamma$.
\end{definition}

\theoremstyle{definition}
\begin{definition}{Monotone Access Structure.}
    \label{defn:monotone}
    An access structure $\Gamma$ is \textbf{monotone} if $B \in \Gamma$ and $B \subseteq C$ implies $C \in \Gamma$.
\end{definition}

\defref{defn:monotone} is particularly useful to us. Essentially, if some subset of participants satisfies the access structure, then all supersets of that subset should also satisfy the access structure. For the task of secret sharing, this definition is intuitive, if not necessary. Indeed, it would be difficult to imagine a procedure that implements an access structure that is \textit{not} monotone. Later on, we will see that the property of monotonicity is not only necessary for the realizability of an access structure, but it is also one of two sufficient conditions.

\theoremstyle{definition}
\begin{definition}{Minimal Access Structure.}
    \label{defn:minimal}
    An access structure $\Gamma$ is \textbf{minimal} if $A \in \Gamma$ implies for every $A' \in \Gamma \setminus \{A\}$, $A' \not\subset A$. Another name for this is an access structure's \textbf{basis}.
\end{definition}

Observe that any monotone access structure has a \textbf{unique minimal access structure}. If we assume that all of the access structures we are talking about are monotone, then we can simplify our discussion by also assuming that we are working only with minimal access structures.

\theoremstyle{definition}
\begin{definition}{Threshold Scheme.}
    \label{defn:threshold-scheme}
    A \textbf{$(t,n)$-threshold secret sharing scheme}, or just "threshold scheme" is a secret sharing scheme among $n$ players such that at least $t$ of those players must combine their respective shares in order to access the secret. The access structure for this scheme is composed of every subset of $\mathcal{P}$ of size $t$. This is also referred to as the set of all $t$-subsets of $n$.
\end{definition}

So, for our $(t,n)$ threshold scheme defined above, the \textit{access structure} can be defined more formally as such: $\Gamma = \{A | A \subseteq \mathcal{P} , |A| = t\}$.

\begin{example}
    Take the set of participants $\mathcal{P} = \{p_1,p_2,p_3\}$ might be $\Gamma = \{p_1p_2,p_2p_3,p_3p_1\}$. This access structure describes a $(2,3)$-threshold scheme.
\end{example}

% do I need to define perfect?
As we mentioned above, Shamir developed one of the first implementations for a perfect threshold scheme based on polynomial interpolation \cite{shamir}. His scheme is a $(t,n)$-threshold scheme. The scheme works as so. First, encode the secret as some number. Then, randomly generate a $(t-1)$-degree polynomial such that the constant term is number which encodes the secret. For each of the $n$ players, generate and distribute one of the pairs $i, p(i)$, where $i \in [1, \cdots, n]$. Each pair acts as a player's "share" of the secret. If $t$ or more players get together and pool together their shares, they can reconstruct the unique $(t-1)-$degree polynomial $p$ that generated those pairs, and then $p(0)$ reveals the secret. Note that having only $t-1$ or fewer pairs gives no information about the secret, because there would be infinitely many polynomials of degree $t-1$ passing through those $t-1$ or fewer pairs.

\section{Graphs and Hypergraphs Preliminaries}

Access structures lend themselves naturally to graphical representations. Let's define some common terms from graph theory, as they will become useful later in this paper. 

\begin{definition}{Graph.}
    \label{defn:graph}
    A \textbf{graph} $G=(V,E)$ is composed of a set of vertices and a set of edges. Each edge is incident to two vertices.
\end{definition}

\begin{definition}{Complement.}
    \label{defn:complement}
    The \textbf{complement} of a graph $G = (V,E)$ is the graph $G^c = (V,E^c)$. The set of vertices in each graph is the same. If two vertices in $G$ are not adjacent, then they are adjacent in $G^c$, and vice-versa.
\end{definition}

\begin{definition}{Line Graph.}
    \label{defn:line-graph}
    The \textbf{line graph} $L(G)$ of a graph $G$ is the graph where each vertex of $L(G)$ corresponds to one edge in $G$. Two vertices in $L(G)$ are adjacent if they are incident to the same vertex in $G$. 
\end{definition}

\begin{definition}{Chromatic Color.}
    \label{defn:colors}
	The \textbf{chromatic color} of a graph $\chi(G)$ is the maximum number of colors needed in order to color the vertices of the graph in such a way that vertices of the same color are not adjacent.
\end{definition}

\begin{definition}{Bipartite Graph.}
    \label{defn:bipartite}
	A \textbf{bipartite graph} is a graph in which the vertices can be separated into two independent and disjoint sets $U,V$. There are no edges between the vertices in any one set of vertices, and every edge in the graph connects a vertex in $U$ with a vertex in $V$.
\end{definition}

\begin{definition}{Complete Graph.}
    \label{defn:clique}
    A \textbf{complete graph} is a graph where each node has an edge with every single other node. A complete graph with $n$ nodes is often denoted as $K_n$. These graphs are also called cliques. A clique with $n$ nodes might be called an $n$-clique.
\end{definition}

\begin{definition}{Hyperedge.}
    \label{defn:hyperedge}
    A \textbf{hyperedge} is an extension of a normal edge. A hyperedge can have any non-zero number of incident vertices.
\end{definition}

\begin{definition}{Hypergraph.}
    \label{defn:hypergraph}
    A \textbf{hypergraph} $H$ is a pair $(V,E)$ that specify a set of vertices $V$ and a set of hyperedges $E$.
\end{definition}

\begin{definition}{Hypercycles.}
    \label{defn:hypercycle}
    A \textbf{hypercycle} $H$ is a hypergraph $H = (V,E)$ where there exists an ordering of hyperedges: $(E_1, E_2, \cdots, E_{m-1})$ such that $E_{i} \cap E_{(i+1) m o d m} \neq \emptyset$ for $i = 1,2,\cdots,m-1$.
\end{definition}

\begin{definition}{Hyperstars.}
    \label{defn:hyperstar}
    A \textbf{hyperstar} is a hypergraph $H = (V,E)$ where the intersection of all the hyperedges $\bigcap_{E_{i} \in E} E_{i} \neq \varnothing$.
\end{definition}

\section{Quantum Computing Preliminaries}

The basic unit of information in a quantum computer is the qubit, or the "quantum bit". A qubit is represented as a \textbf{quantum state}:

\begin{align*}
    \ket{\psi} &= \sum_{i=1}^n \alpha_i\ket{i} \\ 
\end{align*}

The states $\ket{i}$ for $i \in \{1,...,n\}$ represent an orthonormal basis, and the constants $\alpha_1,...,\alpha_n$ are complex numbers normalized such that $\sum_{i=1}^n \alpha_i = 1$.

A \textbf{quantum operation} is a unitary operator $U$ that acts on a quantum state $\ket{\psi}$.

% flesh out this definition
\begin{definition}{Unitary Operator.}
    A \textbf{unitary operator} $U;\mathcal{H} \to \mathcal{H}$ is a linear operator on a Hilbert space $\mathcal{H}$ that satisfies:
    
    \begin{align*}
        U^*U = UU^* = I \\ 
    \end{align*}
    
    Where $U^*$ is the adjoint of $U$.
\end{definition}

Examples of some commonly used operators are the 4 Pauli operators:

\begin{align*}
    I &= \begin{pmatrix}
        1 & 0 \\ 
        0 & 1 \\ 
    \end{pmatrix} \\ 
    X &= \begin{pmatrix}
        0 & 1 \\ 
        1 & 0 \\ 
    \end{pmatrix} \\ 
    Y &= \begin{pmatrix}
        0 & i \\ 
        -i & 0 \\ 
    \end{pmatrix} \\ 
    Z &= \begin{pmatrix}
        1 & 0 \\ 
        0 & -1 \\ 
    \end{pmatrix} \\ 
\end{align*}

One of the most important theorems that has a significant effect on quantum computing algorithms is the \textbf{no-cloning theorem}. Here, we present the no-cloning theorem with proof referencing Mermin's 2007 text \cite{merlin}.

\begin{theorem}{No-Cloning Theorem.}
    \label{thm:no-cloning-thm}
    Given an unknown, arbitrary quantum state $\psi$, there is no valid operator $U$ that can create an identical copy of this state. More formally there exists no operator such that $U(\ket{\psi} \ket{0}) = \ket{\psi}\ket{\psi}$.
\end{theorem}

\begin{proof}
    Assume for the sake of contradiction that there is such an operator. Then $U(\ket{\psi} \ket{0}) = \ket{\psi}\ket{\psi}$ and $U(\ket{\phi} \ket{0}) = \ket{\phi}\ket{\phi}$, for arbitrary quantum states $\ket{\psi}, \ket{\phi}$. Then:
    
    \begin{align*}
        U(\alpha \ket{\psi} + \beta \ket{\phi})\otimes\ket{0} &= (\alpha \ket{\psi} + \beta \ket{\phi})\otimes (\alpha \ket{\psi} + \beta \ket{\phi}) \\ 
        &= \alpha^2\braket{\phi|\phi} + \beta^2\braket{\psi|\psi} + \alpha \beta \braket{\phi|\psi} + \alpha \beta \braket{\psi|\phi} \numberthis \label{eqn:no-clone-1}\\ 
    \end{align*}
    
    But by linearity, we also have:
    
    \begin{align*}
        U(\alpha \ket{\psi} + \beta \ket{\phi})\otimes\ket{0} &= \alpha U \ket{\psi}\ket{0} + \beta U \ket{\phi}\ket{0} \\ 
        &= \alpha \ket{\psi}\ket{\psi} + \beta \ket{\phi}\ket{\phi} \numberthis \label{eqn:no-clone-2}\\ 
    \end{align*}
    
    \eqnref{eqn:no-clone-1} and \eqnref{eqn:no-clone-2} can only be the same if one of $\alpha$ or $\beta$ is equal to 0, which contradicts the assumption that $\ket{\psi}, \ket{\phi}$ are arbitrary.
\end{proof}

\theoremstyle{remark}
\begin{remark}
    Note that the theorem statement can also be made replacing $\ket{0}$ with an arbitrary state $\ket{e}$. What is important is that there is no unitary operator that acts as a "general purpose copier". For example, it would be easy to create an operator that can copy a state $\ket{\phi}$ if we know for a fact that the state is either $\ket{0}$ or $\ket{1}$ \cite{merlin}.
\end{remark}

\section{Quantum Secret Sharing}
\label{section:qss}

% quantum secret sharing scheme
A quantum secret sharing scheme is the quantum analogue of a normal (classical) secret sharing scheme. In these schemes, the information to be shared is quantum, and takes the form of a quantum state $\ket{\psi}$. This idea was first introduced by Hillery et al. in 1998 \cite{Hillery_1999}. In their paper, they give an example of a $((2,3))$-threshold quantum secret sharing scheme implemented using GHZ states. This was then extended by the work of Cleve, Gottesman, and Lo \cite{Cleve_1999}. They introduce the idea of a quantum access structure, and propose implementations for a general quantum secret sharing schemes.

\begin{definition}{Quantum Threshold Scheme.}
    \label{defn:qts}
    A quantum threshold scheme (QTS) is a threshold secret sharing scheme applied to a quantum secret. This secret is denoted as $\ket{\psi}$. As in the classical case, we take a quantum secret and divide it into shares to distribute among a set of individuals. We will denote a QTS among $n$ players with a threshold of $t$ as $((t,n))$, using double parentheses to denote that the scheme is quantum.
\end{definition}

As with classical secret sharing, a QTS is a specific class of a quantum secret sharing schemes--one with a very symmetric access structure. As he hinted on earlier, observe that additional restrictions apply to quantum secret sharing schemes that do not apply to classical secret sharing schemes.

\begin{theorem}
    \label{thm:qss-disjoint}
    A QSS scheme exists for a general access structure $\Gamma$ only if for every $A_1, A_2 \in \Gamma$, $A_1 \cap A_2 \neq \emptyset$.
\end{theorem}

\begin{proof}
    Let's assume, for the sake of contradiction, that there exist two disjoint subsets such that each recover the secret. Then, we would have the possibility of two disjoint groups each obtaining their own copy of the secret, in a sense, creating a way to copy a general quantum state. This violates \thmref{thm:no-cloning-thm}.
\end{proof}

\begin{theorem}
    \label{thm:qts}
    A QTS $((t,n))$ is valid only if $t > \frac{n}{2}$.
\end{theorem}

\begin{proof}
    This is a direct result of \thmref{thm:qss-disjoint}. If $t \leq \frac{n}{2}$, then there would exist at least two authorized subsets that are disjoint. 
\end{proof}

Note that this theorem presents only a necessary condition for the existence of a QTS, not a sufficient one. However, we can indeed prove stronger claims. In 2000, Gottesman showed that a quantum secret sharing scheme exists for an access structure $\Gamma$ as long as $\Gamma$ is monotone and \thmref{thm:qss-disjoint} is satisfied \cite{gottesman_theory_2000}:

\begin{theorem}
    \label{thm:monotone-gamma}
    A quantum secret sharing scheme exists for an access structure iff the access structure is monotonic and the no-cloning theorem is not violated.
\end{theorem}

As we can see, the no-cloning theorem imposes our most strict limitation on the existence of QSS schemes, and this can pose practical limitations to implementing secret sharing schemes in quantum computing. To see why, consider a valid access structure that satisfies the no-cloning theorem. In such an access structure, each authorized set must intersect \textbf{every} other authorized set. Or in other terms, each pair of authorized sets must have at least one player in common. This is a pretty unnatural constraint to impose on an access structure, and can limit the usefulness of the procedure.

In the next section, we will explore some of the methods that researches have used in the past to try to realize more quantum secret sharing schemes without violating the no-cloning theorem.

\section{Circumventing the No-Cloning Theorem}

\subsection{Improvable Secret Sharing}

% talk about using a classical key to encrypt a quantum secret
In 2001, Nascimento presented a scheme that uses a mixture of quantum and classical data to implement secret sharing schemes \cite{nascimento_improving_2001}. His idea combines a method of encrypting quantum data using classical data with the idea of \textit{improvable secret sharing schemes}. The hybrid encryption is presented by Mosca, Tapp, and Wolf \cite{mosca2000private}. They use a classical key $K$ that uses $2n$ bits, where $n$ is the number of qubits in the secret. We encode the state $\ket{\Psi}$ as so. Each pair of bits corresponds to a unitary operator performed on the corresponding qubit. A $00$ means that we apply the $I$ operator, $01$ means $X$, $10$ means $Y$ and $11$ means $Z$. We can then apply a classical secret sharing scheme on the classical key $K$, and a quantum secret sharing scheme on the encrypted quantum state $\ket{\Psi '}$.

% talk about improvable scheme here
Nascimento defines an improvable quantum secret sharing scheme as follows:

\begin{definition}{Improvable Secret Sharing Scheme.}
    \label{defn:improvable}
    An improvable secret sharing scheme is a quantum secret sharing scheme that can be implemented with fewer quantum shares than the number of participants in $\mathcal{P}$.
\end{definition}

Let's say that you would like to implement the access structure $\Gamma$. If possible, restrict the access structure $\Gamma$ to a set of players $B \subsetneq \mathcal{P}$. He uses a clever construction where you take the secret $\ket{\psi}$ and encode it using the classical key $K$ and the encryption we describe above to get an encrypted state $\ket{\widetilde{\psi}}$. We use a classical secret sharing scheme to realize $\Gamma$, and a quantum secret sharing scheme to realize $\Gamma |_B$. Now, an authorized set $A \in \Gamma$ can access the key $K$, and the set $A \cap B$ can access the encrypted quantum state. Putting these together allows reconstruction of the original secret $\ket{\psi}$. This scheme only requires $|B|$ quantum shares. While this construction does not immediately lend itself to being able to implement schemes that violate the no-cloning theorem, we will see that it can become useful later.

\subsection{Assisted Quantum Secret Sharing}

% definitions? 
In 2004, Singh and Srikanth introduced a new scheme called the assisted quantum secret sharing scheme. 

\begin{definition}{Assisted Quantum Secret Sharing Scheme.}
    
\end{definition}

Using this scheme, they showed that the only restriction that existed for quantum secret sharing schemes was monotonicity of the access structure. Their scheme uses the idea of \textit{resident shares} and \textit{player shares}. Resident shares reside with the dealer whom we assume to be trustworthy. The player shares are the ones distributed to the players/participants. In this way, we can realize access structures that would otherwise violate the no-cloning theorem if implemented as a normal QSS scheme. 

For the sake of visualization, Singh and Srikanth depict their access structure as a graph. They term this graph an \textit{access structure graph} (AS graph), which we define below:

\begin{definition}{Access Structure Graph.}
    \label{defn:access-structure-graph}
    An \textbf{access structure graph} (or AS graph) is a graph $G = (V,E)$ where $V = \Gamma$ and $E = \{(A_i \cap A_j \neq \emptyset) \forall\:i,j,i\neq j\}$.
\end{definition}

\begin{theorem}
    \label{thm:complete-as-graph}
    Let $\Gamma = \{A_1,A_2,\dots,A_r\}$ be the access structure. If $\Gamma$ satisfies the no-cloning theorem, then the AS graph of $\Gamma$ must be a complete graph.
\end{theorem}

However, if we don't have a complete graph, then that is fine. We separate the graph into sets of cliques, or as Singh and Srikanth call them, \textit{partially linked classes}. Each clique represents a subset of $\Gamma$ that satisfies the no-cloning theorem alone. Let's say that we can separate the graph into $\lambda$ partially linked classes. Then we only need $\lambda - 1$ resident shares to realize the original access structure. Actually computing the minimum number of resident shares needed to implement a given assisted quantum threshold scheme is equivalent to the minimum clique-cover problem in graph theory, and is NP-hard.

%spirit of the game here:
We take a couple problems with this approach. The first is that the paper does not address the problem of finding the minimum number of resident shares needed, and so it is not that useful taken by itself. It relies heavily on a dealer with all of the power and all of the information. 

We ask, why not add the dealer as an additional player to every single authorized set, so the dealer has one share for every subset in the access structure. This might not be the most optimal in terms of number of shares in the procedure, but it would be the simplest, and it is the same in spirit as the procedure proposed by Singh and Srikanth. Without presenting an efficient way to find the minimum number of shares that must reside with the dealer, simply arguing that such a minimum exists is trivial and useless. But, there are other problems with this procedure as well, and that is the very idea of involving the dealer at all.

An extreme version of a scheme where the dealer is involved in reconstruction would be a procedure in which a group of players that would like to access the secret go to the dealer together. Then, the dealer checks to see if that group of players is present in the access structure. If it is, it simply returns the quantum state, and if it is not, then it does nothing. There is nothing inherently quantum about the procedure, and in a sense, it goes against the spirit of the game.

\subsection{New Improvable Quantum Secret Sharing Schemes}

In 2016, Bai published two papers. The first presents a generalized information theoretic model for quantum secret sharing, and the second uses that model to analyze access structures \cite{bai_generalized_2016} \cite{bai_quantum_2017}. In 2017, Xu presents an extension to Nascimento's \cite{nascimento_improving_2001} work using Bai's generalized model \cite{xu_new_2017}. Xu is able to implement any secret sharing scheme using this generalized hybrid model, which he calls a New Improving Secret Sharing Scheme. 

In \cref{ch3}, we will explore our own ideas that have the potential to relax the constraint imposed by the no-cloning theorem.
