\chapter{Background}

\section{Classical Secret Sharing}

In this section we will describe secret sharing schemes and talk about some of its properties and definitions. A secret sharing scheme allows some \textit{secret} $\mathcal{S}$ to be divided into shares and distributed by some \textit{dealer} $\mathcal{D}$ to a set of \textit{participants} $\mathcal{P}$. Each secret sharing scheme has a corresponding \textit{access structure} denoted by $\Gamma$. $\Gamma$ defines the set of \textit{authorized subsets} of participants that have access to the secret. A subset of participants $A \in \Gamma$ should be able to reconstruct the original secret, but a a set $B \notin \Gamma$ should have \textbf{no} information about the secret, in the sense that all possible values of the secret are equally likely. Let us present some more formalized definitions:

\theoremstyle{definition}
\begin{definition}{Access Structure.}
    \label{defn:access-structure}
    An \textbf{access structure} is often denoted as $\Gamma$. An access structure specifies the set of all authorized subsets of users that are able to recover the secret.
\end{definition}

\theoremstyle{definition}
\begin{definition}{Monotone Access Structure.}
    \label{defn:monotone}
    An access structure $\Gamma$ is \textbf{monotone} if $B \in \Gamma$ and $B \subseteq C$ implies $C \in \Gamma$.
\end{definition}

\theoremstyle{definition}
\begin{definition}{Minimal Access Structure.}
    \label{defn:minimal}
    An access structure $\Gamma$ is \textbf{minimal} if $A \in \Gamma$ implies for every $A' \in \Gamma \setminus \{A\}$, $A' \not\subset A$. Another name for this is an access structure's \textbf{basis}.
\end{definition}

\defref{defn:monotone} is particularly useful to us. Essentially, if some set of participants satisfies the access structure, then all supersets of that set should also satisfy the access structure. We will see later on that the property of monotonicity for an access structure leads to strong claims within the realizability of quantum secret sharing schemes. For now, and for the sake of simplicity, we will assume that all of the access structures we are talking about are monotone. Therefore, when we describe an access structure, we only need to do so using its \textbf{minimal} access structure or basis. 

\theoremstyle{definition}
\begin{definition}{Threshold Scheme.}
    \label{defn:threshold-scheme}
    A \textbf{$(t,n)$-threshold secret sharing scheme} is a secret sharing scheme among $n$ individuals such that at least $t$ of those individuals must work together to access the secret. The access structure for this scheme is composed of every subset of $\mathcal{P}$ of size $t$.
\end{definition}

We say that $A$ is authorized if $A \in \Gamma$. So, for our $(t,n)$ threshold scheme defined above, the access structure can be defined more formally as such: $\Gamma = \{A | A \subseteq \mathcal{P} , |A| = t\}$.

A more concrete example of an access structure for a set of participants $\mathcal{P} = \{p_1,p_2,p_3\}$ might be $\Gamma = \{p_1p_2,p_2p_3,p_3p_1\}$. This access structure describes a $(2,3)$ threshold scheme.

As we mentioned above, Shamir developed one of the first implementations for a perfect threshold scheme based on polynomial interpolation \cite{shamir}. His scheme is a $(t,n)$-threshold secret sharing scheme. The scheme works as so. First, encode the secret as some number. Then, generate a $t-1$ degree polynomial. For each of the $n$ participants, generate a pair $i, p(i)$, where $i \in [1, \cdots, n]$, and distribute one such pair to each participants. Then, if $t$ or more participants get together and share their points, they can reconstruct the unique polynomial $p$ that generated those points, and $p(0)$ reveals the secret. Note that, having only $t-1$ or fewer points gives no information about the secret, because there would be infinitely many polynomials of degree $t-1$ passing through those number of points.

\section{Quantum Computing Preliminaries}

The basic unit of information in a quantum computer is the qubit, or the "quantum bit". A qubit is represented as a \textbf{quantum state}:

% flesh out this definition
\begin{align*}
    \ket{\psi} &= \sum_{i=1}^n \alpha_i\ket{i} \\ 
\end{align*}

A \textbf{quantum operation} is a unitary operator $U$ that acts on a quantum state $\ket{\psi}$.

% flesh out this definition
\begin{definition}{Unitary Operator.}
    A \textbf{unitary operator} $U;\mathcal{H} \to \mathcal{H}$ is a linear operator on a Hilbert space $\mathcal{H}$ that satisfies:
    
    \begin{align*}
        U^*U = UU^* = I \\ 
    \end{align*}
    
    Where $U^*$ is the adjoint of $U$.
\end{definition}

Examples of some commonly used operators are the 4 Pauli operators:

\begin{align*}
    I &= \begin{pmatrix}
        1 & 0 \\ 
        0 & 1 \\ 
    \end{pmatrix} \\ 
    X &= \begin{pmatrix}
        0 & 1 \\ 
        1 & 0 \\ 
    \end{pmatrix} \\ 
    Y &= \begin{pmatrix}
        0 & i \\ 
        -i & 0 \\ 
    \end{pmatrix} \\ 
    Z &= \begin{pmatrix}
        1 & 0 \\ 
        0 & -1 \\ 
    \end{pmatrix} \\ 
\end{align*}

One of the most important theorems that has a large effect on quantum computing algorithms is the \textbf{no-cloning theorem}. Here, we present the no-cloning theorem with proof referencing Mermin's 2007 text \cite{merlin}.

\begin{theorem}{No-Cloning Theorem.}
    \label{thm:no-cloning-thm}
    Given an unknown, arbitrary quantum state $\psi$, there is no valid operator $U$ that can create an identical copy of this state. More formally there exists no operator such that $U(\ket{\psi} \ket{0}) = \ket{\psi}\ket{\psi}$.
\end{theorem}

\begin{proof}
    Assume for the sake of contradiction that there is such an operator. Then $U(\ket{\psi} \ket{0}) = \ket{\psi}\ket{\psi}$ and $U(\ket{\phi} \ket{0}) = \ket{\phi}\ket{\phi}$, for arbitrary quantum states $\ket{\psi}, \ket{\phi}$. Then:
    
    \begin{align*}
        U(\alpha \ket{\psi} + \beta \ket{\phi})\otimes\ket{0} &= (\alpha \ket{\psi} + \beta \ket{\phi})\otimes (\alpha \ket{\psi} + \beta \ket{\phi}) \\ 
        &= \alpha^2\braket{\phi|\phi} + \beta^2\braket{\psi|\psi} + \alpha \beta \braket{\phi|\psi} + \alpha \beta \braket{\psi|\phi} \numberthis \label{eqn:no-clone-1}\\ 
    \end{align*}
    
    But by linearity, we also have:
    
    \begin{align*}
        U(\alpha \ket{\psi} + \beta \ket{\phi})\otimes\ket{0} &= \alpha U \ket{\psi}\ket{0} + \beta U \ket{\phi}\ket{0} \\ 
        &= \alpha \ket{\psi}\ket{\psi} + \beta \ket{\phi}\ket{\phi} \numberthis \label{eqn:no-clone-2}\\ 
    \end{align*}
    
    \eqnref{eqn:no-clone-1} and \eqnref{eqn:no-clone-2} can only be the same if one of $\alpha$ or $\beta$ is equal to 0, which contradicts the assumption that $\ket{\psi}, \ket{\phi}$ are arbitrary.
\end{proof}

\theoremstyle{remark}
\begin{remark}
    Note that the theorem statement can also be made with any state $\ket{e}$ instead of $\ket{0}$. What is important is that there is no unitary operator that acts as a "general purpose copier", where the relation between the state that is to be copied and the state that is being overwritten is arbitrary. For example, it would be easy to create an operator that can copy a state $\ket{\phi}$ if we know for a fact that the state is either $\ket{0}$ or $\ket{1}$ \cite{merlin}.
\end{remark}

\section{Graphs and Hypergraphs Preliminaries}

Access structures lend themselves naturally to graphical representations. Let's define some common terms from graph theory, as they will become useful later in this paper. 

\begin{definition}{Graph.}
    \label{defn:graph}
    A \textbf{graph} $G=(V,E)$ is composed of a set of vertices and a set of edges. Each edge is incident to two vertices.
\end{definition}

\begin{definition}{Complement.}
    \label{defn:complement}
    The \textbf{complement} of a graph $G = (V,E)$ is the graph $G^c = (V,E^c)$. The set of vertices in each graph is the same. If two vertices in $G$ are not adjacent, then they are adjacent in $G^c$, and vice-versa.
\end{definition}

\begin{definition}{Line Graph.}
    \label{defn:line-graph}
    The \textbf{line graph} $L(G)$ of a graph $G$ is the graph where each vertex of $L(G)$ corresponds to one edge in $G$. Two vertices in $L(G)$ are adjacent if they are incident to the same vertex in $G$. 
\end{definition}

\begin{definition}{Chromatic Color.}
    \label{defn:colors}
	The \textbf{chromatic color} of a graph $\chi(G)$ is the maximum number of colors needed in order to color the vertices of the graph in such a way that vertices of the same color are not adjacent.
\end{definition}

\begin{definition}{Bipartite Graph.}
    \label{defn:bipartite}
	A \textbf{bipartite graph} is a graph in which the vertices can be separated into two independent and disjoint sets $U,V$. There are no edges between the vertices in any one set of vertices, and every edge in the graph connects a vertex in $U$ with a vertex in $V$.
\end{definition}

\begin{definition}{Complete Graph.}
    \label{defn:clique}
    A \textbf{complete graph} is a graph where each node has an edge with every single other node. A complete graph with $n$ nodes is often denoted as $K_n$. These graphs are also called cliques. A clique with $n$ nodes might be called an $n$-clique.
\end{definition}

\begin{definition}{Hyperedge.}
    \label{defn:hyperedge}
    A \textbf{hyperedge} is an extension of a normal edge. A hyperedge can have any non-zero number of incident vertices.
\end{definition}

\begin{definition}{Hypergraph.}
    \label{defn:hypergraph}
    A \textbf{hypergraph} $H$ is a pair $(V,E)$ that specify a set of vertices $V$ and a set of hyperedges $E$.
\end{definition}

\begin{definition}{Hypercycles.}
    \label{defn:hypercycle}
    A \textbf{hypercycle} $H$ is a hypergraph $H = (V,E)$ where there exists an ordering of hyperedges: $(E_1, E_2, \cdots, E_{m-1})$ such that $E_{i} \cap E_{(i+1) m o d m} \neq \emptyset$ for $i = 1,2,\cdots,m-1$.
\end{definition}

\begin{definition}{Hyperstars.}
    \label{defn:hyperstar}
    A \textbf{hyperstar} is a hypergraph $H = (V,E)$ where the intersection of all the hyperedges $\bigcap_{E_{i} \in E} E_{i} \neq \varnothing$.
\end{definition}

\section{Quantum Secret Sharing}

% quantum secret sharing scheme
A quantum secret sharing scheme is the quantum analogue of a (classical) secret sharing scheme. In these schemes, the secret to be shared takes the form of a quantum state $\ket{\psi}$. This idea was first introduced by Hillery et al. in 1998 \cite{Hillery_1999}. In their paper, they give an example of a $((2,3))$-threshold quantum secret sharing scheme implemented using GHZ states. This was then extended by the work of Cleve, Gottesman, and Lo \cite{Cleve_1999}. In their paper, they introduce the idea of a quantum access structure, and propose implementation for a general threshold schemes.

\begin{definition}{Quantum Threshold Scheme.}
    \label{defn:qts}
    A quantum threshold scheme (QTS) is a threshold secret sharing scheme applied to a quantum secret. This secret is denoted as $\ket{\psi}$. As in the classical case, we take a quantum secret and divide it into shares to distribute among a set of individuals. We will denote a QTS among $n$ individuals with a threshold of $t$ as $((t,n))$, using double parentheses to denote that the scheme is quantum.
\end{definition}

As with classical secret sharing, quantum threshold schemes are a specific class of quantum secret sharing schemes--ones with a certain access structure. Note that additional restrictions apply to secret sharing schemes applied to quantum information.

\begin{theorem}
    \label{thm:qss-disjoint}
    A QSS scheme exists for a general access structure $\Gamma$ only if for every $A_1, A_2 \in \Gamma$, $A_1 \cap A_2 \neq \emptyset$.
\end{theorem}

\begin{proof}
    Let's assume, for the sake of contradiction, that there exist two disjoint subsets such that each recover the secret. Then, we would have two copies of the quantum state, which violates \thmref{thm:no-cloning-thm}. 
\end{proof}

\begin{theorem}
    \label{thm:qts}
    A QTS $((t,n))$ exists only if $t > \frac{n}{2}$.
\end{theorem}

\begin{proof}
    This theorem is a direct result of \thmref{thm:qss-disjoint}. If $t \leq \frac{n}{2}$, then there would exist at least two authorized subsets that are disjoint. 
\end{proof}

Note that this theorem does not guarantee the existence of a QTS if this inequality of the threshold is met, only that one cannot exist if the inequality is not met. However, we can indeed prove stronger claims. In 2000, Gottesman showed that a quantum secret sharing scheme exists for an access structure $\Gamma$ as long as $\Gamma$ is monotone and \thmref{thm:qss-disjoint} is satisfied \cite{gottesman_theory_2000}:

\begin{theorem}
    \label{thm:monotone-gamma}
    A quantum secret sharing scheme exists for an access structure iff the access structure is monotonic and the no-cloning theorem is not violated.
\end{theorem}

As we can see, the no-cloning theorem imposes most of the limitations on realizable QSS schemes. This can actually be a pretty serious issue in terms of the applicability of QSS schemes. To see why, think about what an access structure that satisfies the no-cloning theorem actually looks like. Each authorized set must intersect \textbf{every} other authorized set. Or in other terms, each pair of authorized sets must have at least one individual in common. This is not exactly a property that you would expect in a group where secret sharing is useful. It is much more common to have situations where there is more trust within a group of people, and almost no trust with people outside of that group.

In the next section, we will explore some of the methods that researches have used in the past to try to realize secret sharing schemes without violating the no-cloning theorem.

\section{Circumventing the No-Cloning Theorem}

\subsection{Improvable Secret Sharing}

% talk about using a classical key to encrypt a quantum secret
In 2001, Nascimento presented a scheme that uses a mixture of quantum and classical data to implement secret sharing schemes \cite{nascimento_improving_2001}. His idea combines a method of encrypting quantum data using classical data with the idea of \textit{improvable secret sharing schemes}. The hybrid encryption is presented by Mosca, Tapp, and Wolf \cite{mosca2000private}. They use a classical key $K$ that uses $2n$ bits, where $n$ is the number of qubits in the secret. We encode the state $\ket{\Psi}$ as so. Each pair of bits corresponds to a unitary operator performed on the corresponding qubit. A $00$ means that we apply the $I$ operator, $01$ means $X$, $10$ means $Y$ and $11$ means $Z$. We can then apply a classical secret sharing scheme on the classical key $K$, and a quantum secret sharing scheme on the encrypted quantum state $\ket{\Psi '}$.

% talk about improvable scheme here
Nascimento defines an improvable quantum secret sharing scheme as follows:

\begin{definition}{Improvable Secret Sharing Scheme.}
    \label{defn:improvable}
    An improvable secret sharing scheme is a quantum secret sharing scheme that can be implemented with fewer quantum shares than the number of participants in $\mathcal{P}$.
\end{definition}

Let's say that you would like to implement the access structure $\Gamma$. If possible, restrict the access structure $\Gamma$ to a set of players $B \subsetneq \mathcal{P}$. He uses a clever construction where you take the secret $\ket{\psi}$ and encode it using the classical key $K$ and the encryption we describe above to get an encrypted state $\ket{\widetilde{\psi}}$. We use a classical secret sharing scheme to realize $\Gamma$, and a quantum secret sharing scheme to realize $\Gamma |_B$. Now, an authorized set $A \in \Gamma$ can access the key $K$, and the set $A \cap B$ can access the encrypted quantum state. Putting these together allows reconstruction of the original secret $\ket{\psi}$. This scheme only requires $|B|$ quantum shares. While this construction does not immediately lend itself to being able to implement schemes that violate the no-cloning theorem, we will see that it can become useful later.

\subsection{Assisted Quantum Secret Sharing}

% definitions? 
In 2004, Singh and Srikanth introduced a new scheme called the assisted quantum secret sharing scheme. 

\begin{definition}{Assisted Quantum Secret Sharing Scheme.}
    
\end{definition}

Using this scheme, they showed that the only restriction that existed for quantum secret sharing schemes was monotonicity of the access structure. Their scheme uses the idea of \textit{resident shares} and \textit{player shares}. Resident shares reside with the dealer whom we assume to be trustworthy. The player shares are the ones distributed to the players/participants. In this way, we can realize access structures that would otherwise violate the no-cloning theorem if implemented as a normal QSS scheme. 

For the sake of visualization, Singh and Srikanth depict their access structure as a graph. They term this graph an \textit{access structure graph} (AS graph), which we define below:

\begin{definition}{Access Structure Graph.}
    \label{defn:access-structure-graph}
    An \textbf{access structure graph} (or AS graph) is a graph $G = (V,E)$ where $V = \Gamma$ and $E = \{(A_i \cap A_j \neq \emptyset) \forall\:i,j,i\neq j\}$.
\end{definition}

\begin{theorem}
    \label{thm:complete-as-graph}
    Let $\Gamma = \{A_1,A_2,\dots,A_r\}$ be the access structure. If $\Gamma$ satisfies the no-cloning theorem, then the AS graph of $\Gamma$ must be a complete graph.
\end{theorem}

However, if we don't have a complete graph, then that is fine. We separate the graph into a set of cliques, or as Singh and Srikanth call it, \textit{partially linked classes}. Each clique represents a subset of $\Gamma$ that satisfies the no-cloning theorem alone. Let's say that we can separate the graph into $\lambda$ partially linked classes. Then we only need $\lambda - 1$ resident shares to realize the original access structure.

\subsection{Compressing Secret Sharing Scheme}

\subsection{Inflating Secret Sharing Scheme}

\subsection{Twin-Threhold}



\subsection{New Improvable Quantum Secret Sharing Schemes}

In 2016, Bai published two papers. The first presents a generalized information theoretic model for quantum secret sharing, and the second uses that model to analyze access structures \cite{bai_generalized_2016} \cite{bai_quantum_2017}. In 2017, Xu presents an extension to Nascimento's \cite{nascimento_improving_2001} work using Bai's generalized model \cite{xu_new_2017}. Xu is able to implement any secret sharing scheme using this generalized hybrid model, which he calls a New Improving Secret Sharing Scheme. 

In \cref{ch3}, we will explore our own ideas that have the potential to relax the constraint imposed by the no-cloning theorem.
