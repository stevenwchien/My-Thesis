\chapter{Secret Sharing}
\label{ch:ss}

In this chapter we will describe secret sharing schemes and talk about some of its properties and definitions. Then, we will extend our discussion to quantum secret sharing schemes, and present some preliminary definitions and results.

\section{Classical Secret Sharing}
\label{sec:css}

 A secret sharing scheme allows some \textbf{secret} $\mathcal{S}$ to be divided into \textbf{shares} and distributed by some \textbf{dealer} $\mathcal{D}$ to a set of \textbf{participants} or \textbf{players} $\mathcal{P}$. Each secret sharing scheme has a corresponding \textbf{access structure} denoted by $\Gamma$. $\Gamma$ defines the set of \textbf{authorized subsets} of $\mathcal{P}$ that have access to the secret. A subset of participants $A \in \Gamma$ should be able to reconstruct the original secret in its entirety, but a a set $B \notin \Gamma$ should have \textbf{no information} about the secret, in the sense that all possible values of the secret are equally likely. It is this final condition that makes implementing secret sharing schemes more difficult and interesting than just, say, taking a string and dividing it into equal chunks and giving one chunk to each person. Let us present some more formalized definitions:

\begin{definition}{Access Structure.}
    \label{defn:access-structure}
    An \textbf{access structure} is often denoted as $\Gamma$. An access structure specifies the set of all authorized subsets of players that are able to recover the secret.
\end{definition}

\begin{definition}{Authorized Set.}
    \label{defn:authorized-set}
    An \textbf{authorized set} is a subset $A \subseteq \mathcal{P}$ such that $A \in \Gamma$ for an access structure $\Gamma$. The participants in an authorized set $A$ have the ability to access the secret together.
\end{definition}

\begin{definition}{Monotone Access Structure.}
    \label{defn:monotone}
    An access structure $\Gamma$ is \textbf{monotone} if $B \in \Gamma$ and $B \subseteq C$ implies $C \in \Gamma$.
\end{definition}

\defref{defn:monotone} is particularly useful to us. Essentially, if some subset of participants is in the access structure, then all supersets of that subset should also be in the access structure. For the task of secret sharing, this definition is intuitive, if not necessary. Indeed, it would be difficult to conceive of a secret sharing procedure that implements an access structure that is \textit{not} monotone. Later on, we will see that the property of monotonicity is not only necessary for the realizability of quantum access structures, but it is also one of two sufficient conditions.

\begin{definition}{Minimal Access Structure.}
    \label{defn:minimal-as}
    An access structure $\Gamma$ is \textbf{minimal} if $A \in \Gamma$ implies for every $A' \in \Gamma \setminus \{A\}$, $A' \not\subset A$. Another name for this is an access structure's \textbf{basis}.
\end{definition}

\begin{definition}{Minimal Authorized Set.}
    \label{defn:minimal-authorized-set}
    A \textbf{minimal authorized set} is simply one of the authorized sets in the minimal access structure.
\end{definition}

Observe that any monotone access structure has a \textbf{unique minimal access structure}. If we assume that all of the access structures we are talking about are monotone, then this gives us a much simpler way to analyze access structures.

\begin{example}
    Let $\mathcal{P} = \{A,B,C,D\}$. Let $\Gamma_1 = \{(A,B), (A,C), (A,D)\}$. Let $\Gamma_2 = \{(A,B), (A,C), (A,D), (A,B,C)\}$. Assuming both access structures are monotone, then $\Gamma_1$ and $\Gamma_2$ represent the same access structure, because $(A,C) \subset (A,B,C)$. However, $\Gamma_1$ is a minimal access structure, and in fact, it is the \textbf{unique} minimal access structure of $\Gamma_2$. We say that $\Gamma_1$ and $\Gamma_2$ each have 3 minimal authorized sets.
\end{example}

In a way, we can think of the minimal authorized sets as the ``smallest common denominators'' which make up an access structure. We do not need to include an authorized set $(A,B,C)$ because it is implied to be authorized if $(A,B)$ is already a part of the access structure.

\begin{definition}{Threshold Scheme.}
    \label{defn:threshold-scheme}
    A \textbf{$(t,n)$-threshold secret sharing scheme}, or just ``threshold scheme'' is a secret sharing scheme among $n$ players such that at least $t$ of those players must combine their respective shares in order to access the secret. The access structure for this scheme is composed of every subset of $\mathcal{P}$ of size $t$. This is also referred to as the set of all $t$-subsets of $n$.
\end{definition}

So, for our $(t,n)$ threshold scheme defined above, the \textit{access structure} can be defined more formally as such: $\Gamma = \{A | A \subseteq \mathcal{P} , |A| = t\}$.

\begin{example}
    Take the set of participants $\mathcal{P} = \{p_1,p_2,p_3\}$. Let $\Gamma = \{p_1p_2,p_2p_3,p_3p_1\}$. This access structure describes a $(2,3)$-threshold scheme. At least 2 out of the 3 players are needed to reconstruct the secret.
\end{example}

As we mentioned above, Shamir developed one of the first implementations for a perfect threshold scheme based on polynomial interpolation \cite{shamir_how_1979}. A perfect secret sharing scheme is defined as one where authorized subsets have access to the secret, and unauthorized subsets have no information at all, in an information-theoretical sense, of the secret. Despite their shares, all possible values of the secret are possible.

Shamir's scheme is a $(t,n)$-threshold scheme. The scheme works as so. First, encode the secret as a number such that the secret is retrievable in its original form. Then, randomly generate a $(t-1)$th-degree polynomial such that the constant term is the number encoding the secret. For each of the $n$ players, generate and distribute one of the pairs $(i, p(i))$, where $i \in [1, \cdots, n]$. Each pair of numbers acts as a player's share of the secret. If $t$ or more players work together and pool their shares, they can reconstruct the unique $(t-1)$th-degree polynomial $p$ that generated those pairs, and then $p(0)$ reveals the secret. Having only $t-1$ or fewer pairs gives no information about the secret, because there would be infinitely many polynomials of degree $t-1$ passing through those $t-1$ or fewer pairs, and all values of the secret would be equally likely.

\section{Quantum Secret Sharing}
\label{section:qss}

A quantum secret sharing scheme is the quantum analogue of a classical secret sharing scheme in the sense that the information to be shared is quantum. This idea was first introduced by Hillery et al. in 1999 \cite{hillery_quantum_1999}. In their paper, they give an example of a $((2,3))$-threshold quantum secret sharing scheme implemented using Greenberger-Horne-Zeilinger (GHZ) states. This was then extended by the work of Cleve, Gottesman, and Lo \cite{cleve_how_1999}. They introduce the idea of a quantum access structure, and propose implementations for general quantum secret sharing schemes. We start off by extending some definitions from the previous section.

\begin{definition}{Quantum Threshold Scheme.}
    \label{defn:qts}
    A $((t,n))$-quantum threshold scheme (QTS) is a $(t,n)$-threshold secret sharing scheme applied to a quantum secret. We use double parentheses to illustrate that the scheme is quantum. The quantum secret is denoted $\ket{\psi}$. As in the classical case, we take a quantum secret and divide it into $n$ shares to distribute among a set of participants. We require at least $t$ of those participants to combine their shares in order to recover the quantum state $\ket{\psi}$.
\end{definition}

As with classical secret sharing, a QTS is a specific class of a quantum secret sharing schemes -- one with a particular access structure. All of the other definitions from classical secret sharing are applicable in the quantum case. However,  additional restrictions apply to quantum secret sharing schemes that do not apply to their classical counterparts.

\begin{theorem}
    \label{thm:qss-disjoint}
    A QSS scheme exists for a general access structure $\Gamma$ only if for every $A_1, A_2 \in \Gamma$, $A_1 \cap A_2 \neq \emptyset$.
\end{theorem}

\begin{proof}
    Let us assume, for the sake of contradiction, that there exists a scheme where two disjoint subsets of players can each recover the secret quantum state. Then, we would have the possibility of two disjoint groups each obtaining their own copy of the secret. In a sense, this creates a way to copy a general quantum state. This is not possible because it violates the no-cloning theorem, so this scheme could not exist.
\end{proof}

\begin{corollary}
    \label{cor:qts}
    A QTS $((t,n))$ is valid only if $t > \frac{n}{2}$.
\end{corollary}

\begin{proof}
    This is a direct result of \thmref{thm:qss-disjoint}. If $t \leq \frac{n}{2}$, then there would exist at least two authorized subsets that are disjoint. 
\end{proof}

Note that this theorem presents only a necessary condition for the existence of a QTS, not a sufficient one. However, we can indeed prove stronger claims. In 2000, Gottesman showed that a quantum secret sharing scheme exists for an access structure $\Gamma$ as long as $\Gamma$ is monotone and \thmref{thm:qss-disjoint} is satisfied \cite{gottesman_theory_2000}:

\begin{theorem}
    \label{thm:monotone-gamma}
    A quantum secret sharing scheme exists for an access structure iff the access structure is monotone and the no-cloning theorem is not violated.
\end{theorem}

In later discussion, we are going to call any QSS scheme that satisfies the conditions of \thmref{thm:monotone-gamma} valid. As we can see, the no-cloning theorem imposes our most strict limitation on the existence of QSS schemes, and this can pose practical limitations to implementing secret sharing schemes in quantum computing. To see why, consider a valid access structure that satisfies the no-cloning theorem. In such an access structure, each authorized set must intersect \textbf{every} other authorized set. Or in other terms, each pair of authorized sets must have at least one player in common. This is a somewhat unnatural constraint to impose on an access structure, and can limit the usefulness of the procedure.

\section{A Quantum Information Theoretical Model for Quantum Secret Sharing}

When discussing quantum cryptography, quantum information theory can be a very important tool. It will allow us to provide a precise definition for quantum secret sharing and make rigorous claims. Relative to the amount of literature on quantum secret sharing schemes, previous work on this kind of approach to quantum secret sharing is rather limited.

Imai et al. were one of the first to propose an information theoretical model for quantum secret sharing \cite{imai_quantum_2003}. Later, Rietjens applies this definition in the analysis of a number of different quantum secret sharing implementations \cite{rietjens_quantum_2005}. More recently, Bai et al. generalize Imai's initial definition so that it can apply to more quantum secret sharing schemes. However, this comes at the cost of schemes that are potentially less secure. First, we present the definition from \cite{imai_quantum_2003}:

Let $S$ be the quantum system of the secret. The system $S$ can take on the possible states: $\{\ket{\psi_i}\}$ each with probability $p_i$. The density operator representing the secret is then:
\begin{align}
    \rho_s &= \sum_i p_i \ket{\psi_i} \bra{\psi_i}
\end{align}
Then $\mathcal{P} = \{P_1,P_2,...,P_n\}$ is the set of players, where we will take $P_i$ to represent both the $i$-th player \textbf{and} the shares in possession of $P_i$, for notational simplicity. Similarly, $A \subseteq \mathcal{P}$ will both represent a subset of the players $\mathcal{P}$ and also the set of shares in the possession of that subset of players. We let $R$ be a reference system such that the composite system $RS$ is in a pure state.

A quantum secret sharing scheme is a \textbf{completely positive map} $\Lambda$ (maps positive elements to positive elements) such that:
\begin{align}
    \Lambda_D: S(\mathcal{H_S}) \to S(\mathcal{H}_1 \otimes \dots \otimes \mathcal{H}_n)
\end{align}
Where $\mathcal{H}_i$ is the Hilbert space containing the shares of player $i$, and $S(\mathcal{H}_A)$ is the state space of quantum system $A$. After the operation, the state $\ket{RS}$ becomes $\ket{RP_1P_2 \dots P_n}$.

\begin{definition}{Perfect Quantum Secret Sharing Scheme.}
    \label{defn:perfect-qss}
    A \textbf{perfect quantum secret sharing scheme} is one that satisfies the following conditions:
    \begin{enumerate}
        \item Recoverability requirement: For all $A \in \Gamma$, there exists a recovery map $T_A: \mathcal{H}_A \to \mathcal{H}_S$ such that 
        \begin{align} 
            \rho^{RA} &\to \ket{RS} \\
            I(R:A) &= I(R:S)
        \end{align}
        \item Secrecy requirement: For all $B \notin \Gamma$ we have that:
        \begin{align}
            I(R:B) = 0
        \end{align}
    \end{enumerate}
    Intuitively, this means that the quantum system of the shares in the possession of any authorized set should have the same information content as the quantum system of the secret, and the quantum system of any unauthorized set has no information at all about the secret.
\end{definition}

We make sure to indicate that this scheme is \textit{perfect}, because it is also possible to create schemes that are imperfect. Bai et al. refer to this as a \textbf{generalized information theoretical model}, meaning unauthorized sets do have some amount of information about the secret. Using similar notation, the above formulation, $I(R:B) < I(R:S)$, but $I(R:B) \neq 0$ \cite{bai_generalized_2016}. Naturally, relaxing this constraint increases the number of schemes that satisfy the definition. For the purposes of our thesis, we will stick with the definition proposed by Imai.

The following theorem characterizes schemes that are known to be perfect. This will help us later in analyzing the security of our implementation.

\begin{theorem}
    A quantum secret sharing scheme in which the unauthorized sets are exactly the complements of the authorized sets is perfect. Specifically, this is a scheme where:
    \begin{align}
        A \in \Gamma \to A^c \notin \Gamma \\ 
        B \notin \Gamma \to B^c \in \Gamma
    \end{align}
\end{theorem}

Gottesman first gave a proof for this, and referred to them as \textbf{maximal} access structures \cite{gottesman_theory_2000}. This name comes from the fact we cannot add anymore sets into a maximal access structure $\Gamma$ without violating the no-cloning theorem. Suppose that we did try to add a set into $\Gamma$. Then by the definition of maximal, we must have added a set to $\Gamma$ whose complement was already in $\Gamma$. This violates the no-cloning theorem.

\begin{remark}
    An access structure can be both minimal and maximal at the same time.
\end{remark}

\section{Circumventing the No-Cloning Theorem}

\subsection{Assisted Quantum Secret Sharing}
\label{ssec:aqss}

\thmref{thm:monotone-gamma} shows us that the no-cloning theorem imposes the strictest bounds on the number of realizable quantum secret sharing schemes. In 2004, Singh and Srikanth introduced a new scheme called the \textbf{assisted quantum secret sharing scheme} \cite{singh_assisted_2004}. 

\begin{definition}{Assisted Quantum Secret Sharing Scheme.}
    \label{defn:aqss}
    This is a quantum secret sharing procedure which involves the dealer in reconstruction. The dealer retains several shares in its own possession, called \textit{resident shares}. All other shares are called \textit{player shares}.
\end{definition}

What is significant here is that the dealer must be trustworthy, and reconstruction of the secret must be done by the dealer. In their paper, they introduce a graphical representation of the access structure to aid in their discussion. They call this graph an \textit{access structure graph} (AS graph), which we define formally below:

\begin{definition}{Access Structure Graph.}
    \label{defn:access-structure-graph}
    An \textbf{access structure graph} (or AS graph) is a graph $G = (V,E)$ where $V = \Gamma$ and $E = \{(A_i \cap A_j \neq \emptyset) \forall\:i,j,i\neq j\}$.
\end{definition}

Each authorized set in the access structure corresponds to a vertex in the graph, and there are edges between two vertices if their authorized sets intersect. This graph is useful because we can draw connections between properties of the graph and properties of the access structure, like the one here:

\begin{proposition}
    \label{prop:complete-as-graph}
    Let $\Gamma = \{A_1,A_2,\dots,A_r\}$ be the access structure. $\Gamma$ satisfies the no-cloning theorem iff its AS graph is a complete graph.
\end{proposition}

\begin{proof}
    By \thmref{thm:qss-disjoint}, we know that each pair of authorized subsets must have a nonempty intersection. Each of these intersections correspond to an edge in the AS graph, so there must be an edge between every pair of vertices.
    
    If $\Gamma$ does not satisfy the no-cloning theorem, then we know that there must be at least one pair of authorized sets that do not intersect each other. These corresponding vertices in the AS graph are non-adjacent.
\end{proof}

If the access structure $\Gamma$ does not satisfy the no-cloning theorem, we know that its AS graph is not a complete graph. However, we can still use its AS graph to our advantage. We use the graph to find a clique covering, or as Singh and Srikanth call it, a set of \textit{partially linked classes}. Each clique/class represents a subset of $\Gamma$ that would satisfy the no-cloning theorem alone. Let us say that we separate the graph into $\lambda$ partially linked classes. Then we only need $\lambda - 1$ resident shares to realize the original access structure. 

What we do is we split the original quantum secret using a $((\lambda, 2 \lambda - 1))$ quantum threshold scheme. Of these shares, $\lambda-1$ are dealer shares and stay in the possession of the dealer. The remaining $\lambda$ shares are distributed to the players, one share for each partially linked class. We apply a secret sharing scheme on each of these player shares so that they implement the access structure corresponding to their class. This way, if the players in an authorized set work together, then they can recover a player share. They combine this with the $\lambda-1$ shares kept by the dealer and can recover the original secret. Using this construction, the no-cloning theorem no longer restricts the set of access structures that can be implemented.

A problem that the authors leave open is computing the minimum number of resident shares needed to implement a given access structure. This problem is equivalent to finding the clique cover number of a graph, which is an NP-hard problem in graph theory.

Although this scheme removes the restriction of the no-cloning theorem, we are not entirely satisfied with this approach. As we mentioned above, there is no easy way to find the minimum number of resident shares that are needed to implement an arbitrary access structure.

Additionally, the requirement that the dealer be used for reconstruction in this way is rather unnatural. One might imagine a more exaggerated version of the procedure as follows. The dealer holds onto the quantum secret, and has a list of all of the authorized sets that can access the secret. For any group that would like to reconstruct the secret, the dealer simply checks to see if that group of players is present in the access structure. If it is, it returns the quantum state, and if it is not, then it does nothing. This procedure certainly removes the no-cloning theorem restriction. It may even remove the restriction that access structures need to be monotone. But, it is not really a quantum secret sharing scheme. It goes against the spirit of the game -- there is nothing inherently quantum about the procedure, and in a way, it defeats the purpose of secret sharing.

One final problem with the scheme is if two disjoint authorized sets which are disjoint try to recover the secret, then by the no-cloning theorem they still cannot both obtain the secret. This introduces an unintended and artificial limitation in the scheme. Given an implementation of an access structure, it does not seem right that there is an unwritten condition that if one authorized set recovers the secret, then another authorized set cannot. 

Observe that such a situation would not happen in an access structure that satisfies the no-cloning theorem. Imagine that there are two authorized sets $A_1,A_2$ that would both like to recover the secret. Let's say $A_1$ does so. Because they must have a non-empty intersection ($A_1 \cap A_2 \neq \emptyset$), then those players in the intersection $A_1 \cap A_2$ would then be able to share access to the secret with $A_2$, based on the assumption that the members of $A_2$ have agreed to work together.

In \chref{ch:good-stuff}, we propose a novel approach to relaxing the constraint imposed by the no-cloning theorem.