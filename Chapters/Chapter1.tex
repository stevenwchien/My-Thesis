\chapter{Introduction}

Secret sharing is a procedure by which we take some sensitive data or a "secret" and distribute it among individual parties so that access to the data requires sufficiently large subsets of those parties to come together. One of the first secret sharing schemes was introduced by Shamir [1]. In his paper, he presents a \textit{($k,n$) threshold scheme}, which allows the secret to be easily retrieved if $k$ or more people come together, but $k-1$ or fewer people have no additional information about the secret. An example application of a secret sharing scheme might be in a board of directors vote, where, say, out of 10 people, at least 7 must agree on a vote to take some executive action.

The threshold scheme is a naturally symmetric example of a secret sharing scheme, but it is not the only kind. In general, we can define an \textit{access structure} for a secret sharing scheme, which defines the set of subsets of individuals that are authorized to recover the secret. Not every access structure is possible. One of the main restrictions is that an access structure must be \textit{monotone}. A \textit{monotone access structure} satisfies the condition that any subset of a set not in the access structure is also not in the access structure. This is intuitive. If this were not true, then the larger unauthorized set that contains the authorized one could gain access to the secret simply by using one of its own subsets. 

Quantum computing algorithms, and specifically, quantum cryptography, will become very important in the future, with the current research into implementing quantum computers and quantum information. Many papers have discussed the implementation of secret sharing to quantum information. Specifically, people like Gottesman, Cleveland, Lo, have discussed both threshold schemes and schemes with general access structures. One of the main limitations of these schemes come as a consequence of the \textit{no-cloning theorem}. This theorem follows directly from quantum mechanics. It states that we cannot In this paper, we will first present relevant background for both classical and quantum secret sharing (QSS schemes). We will do things like prove the no-cloning theorem, and establish necessary quantum computing preliminaries. Then, we will reason about quantum threshold schemes (QTS) with multiple copies of the same state, and present our findings in a systematic manner. Finally, we will discuss the implications and possible future work.