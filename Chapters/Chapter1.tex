\chapter{Introduction}

Secret sharing is a procedure by which we take some sensitive data and distribute it among individual parties such that access to the data requires sufficiently large subsets of those parties to come together. Secret sharing can be applied to a voting procedure within a board of directors, or more generally, cases where there is mutual distrust within a group that must be able to cooperate to access some sensitive information. A naturally symmetric and useful version of secret sharing is one in which, out of $n$ people, at least $t$ must come together to gain access to a secret. This secret sharing scheme is called a $(t,n)$ threshold scheme. One of the first methods of implementing such a scheme was proposed by Blakely and Shamir. They used a method where the data is encoded as a number, and the number is used as the constant of a $t-1$-degree polynomial. All $t$ people are given a point $(k, p(k))$ for $n$ real numbers. A group of $t$ people will have $t$ points, and will be able to determine the unique $t-1$ polynomial that passes through those $t$ points. $p(0)$ then gives the secret.

However, threshold schemes are not the only kinds of secret sharing schemes. In general, we can define an \textit{access structure} for a scheme, which defines the set of subsets that are authorized to recover the secret. Not every access structure is possible. One of the main restrictions is that an access structure must be \textit{monotone}. A \textit{monotone access structure} satisfies the condition that any subset of a set not in the access structure is also not in the access structure. This is intuitive. If this were not true, then the larger unauthorized set that contains the authorized one could gain access to the secret simply by using one of its own subsets. 

With the increasing practicality of quantum computing, and the development of algorithms that manipulate quantum information, we find the need for quantum cryptography. Many papers have discussed the implementation of secret sharing to quantum information. Specifically, people like Gottesman, Cleveland, Lo, have discussed both threshold schemes and schemes with general access structures. One of the main limitations of these schemes come as a consequence of the \textit{no-cloning theorem}. This theorem follows directly from quantum mechanics. It states that we cannot In this paper, we will first present relevant background for both classical and quantum secret sharing (QSS schemes. We will do things like prove the no-cloning theorem, and establish necessary quantum computing preliminaries. Then, we will reason about quantum threshold schemes (QTS) with multiple copies of the same state, and present our findings in a systematic manner. Finally, we will discuss the implications and possible future work.