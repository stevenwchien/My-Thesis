\chapter{Introduction}

Picture this: It's Sunday night, 9pm. You are sitting on your couch watching your favorite show while sipping on a Natural Light's Naturdays.

% TODO: cite shamir
Secret sharing is a cryptographic procedure by which we take some sensitive data or a "secret" and distribute it among individual parties in such a way that access to the secret requires sufficiently large subsets of those parties to come together. One of the first secret sharing schemes was introduced by Shamir \cite{shamir}. In his paper, he presents an efficient implementation of a \textit{($t,n$) threshold scheme}, which allows the secret to be easily retrieved if $t$ or more people come together, but $t-1$ or fewer people have no additional information about the secret. Applications of such schemes are numerous, but some examples include giving a group of bank executives  access to a vault, requiring multiple officials to launch nuclear warheads, and voting procedures on a board of directors.

The threshold scheme is a naturally symmetric example of a secret sharing scheme, but it is not the only kind. In general, we can define an \textit{access structure} for a secret sharing scheme, which defines the set of subsets of individuals that are authorized to recover the secret. Not every access structure is possible. One of the main restrictions is that an access structure must be \textit{monotone}. A \textit{monotone access structure} satisfies the condition that any subset of a set not in the access structure is also not in the access structure. This is intuitive. If this were not true, then the larger unauthorized set that contains the authorized one could gain access to the secret simply by using one of its own subsets. 

With the current research in implementing quantum computers and quantum information, quantum computing algorithms and quantum cryptography will become very important in the future. Many papers have discussed the implementation of secret sharing to quantum information. Hillery, Bu\u{z}ek, and Berthiaume presented one of the first schemes that involves using GHZ states to split quantum information into two parts such that both are necessary to recover the original information \cite{Hillery_1999}. Gottesman, Cleve, Lo use quantum error correcting codes to implement both threshold schemes and schemes with general access structures \cite{Cleve_1999}. Smith presents a construction for general access structures using monotone span programs. One of the main limitations of quantum secret sharing schemes come as a consequence of the \textit{no-cloning theorem}. This theorem follows directly from quantum mechanics. It states that we cannot make a copy, or a clone, of an unknown quantum state. 

The goal of this paper is to explore ideas that can increase the number of realizable quantum secret sharing schemes. First, we will present relevant background for both classical and quantum secret sharing (QSS schemes). We will do things like prove the no-cloning theorem, and establish necessary quantum computing preliminaries. We will also bring in some other theory that will become useful in our analysis, such as some definitions and results from graph theory. Then, we will reason about quantum threshold schemes (QTS) using multiple copies of the same state, and present our findings in a systematic manner. Finally, we will discuss the implications and possible future work.