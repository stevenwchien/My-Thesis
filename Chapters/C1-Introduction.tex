\chapter{Introduction}
\label{introduction}

\setstretch{0.5}

\textit{``If computers that you build are quantum,}

\textit{Then spies of all factions will want 'em.}

\textit{Our codes will all fail,}

\textit{And they'll read our email,}

\textit{Till we've crypto that's quantum, and daunt 'em.''}

\quad -- Jennifer and Peter Shor

\setstretch{1.3}
% quantum computing: what a concept
Quantum computing is an exciting and active field of research. Much of the work being done today deals with the implementation of quantum information and quantum hardware. However, the use of quantum computers to store and manipulate information demands secure encryption schemes and robust error correction. This has motivated an expansive amount of literature in the fields of quantum cryptography and quantum error correction (QEC). 

Research in quantum cryptography advances with two main goals. The first is to provide quantum implementations of useful classical cryptographic procedures. This is important if we are to ever use quantum computers to manipulate, store, and transfer sensitive data. The second is to develop schemes that are secure not only against classical computers, but also against quantum computers. One concern is that quantum computers will enable us to break certain classical encryption schemes, and to an extent, this is true. A popular example of a potentially insecure encryption scheme is RSA, whose security depends on the computational complexity of factoring large numbers. The vulnerability comes from the existence of an efficient \textit{quantum} algorithm for factoring integers, invented by Peter Shor \cite{Shor_1997}. 

Error correction is something that we take for granted in today's computers. However, it poses a particularly difficult problem in quantum computers. One challenge is that the physical implementations of quantum information are relatively fragile, and are susceptible to effects like decoherence from external systems. Another challenge is a consequence of the \textit{no-cloning theorem}. This theorem states that we cannot make a copy, or a clone, of an unknown quantum state. Many error correcting schemes in classical computers rely on the ability to copy information, using the extra copies to create redundancy. These obstacles might suggest that error correcting in quantum computers is doomed to fail; surprisingly, this is not the case.

An important cryptographic procedure related to error correction is called secret sharing. The goal of secret sharing is to take sensitive data (the secret) and distribute it among a set of participants in such a way that requires sufficiently large subsets of those participants to work together in order to reconstruct the secret. There are many situations in which these schemes are useful. These situations are often characterized by: information that is too sensitive for just one person to have access; mutual distrust, yet a need to cooperate among members of a group; or the need to distribute power among multiple people. Example applications include giving a group of bank executives access to a vault, requiring multiple officials to launch nuclear warheads, or creating secure voting procedures for a board of directors. For these reasons, developing secure and efficient methods of sharing secrets using quantum computers is an important task. 

When applied to quantum information, these cryptographic procedures are known as \textit{quantum secret sharing} (QSS), and they are a well-studied field in quantum cryptography. In 1999, Hillery, Bu\u{z}ek, and Berthiaume are credited with presenting one of the first schemes that involves using GHZ states to split quantum information into two parts such that both are necessary to recover the original information \cite{hillery_quantum_1999}. Gottesman, Cleve, and Lo use quantum error correcting codes to implement both threshold schemes and schemes with general access structures \cite{cleve_how_1999}. Smith presents a construction for general access structures using monotone span programs \cite{smith_quantum_2000}. Later, Gottesman proves several important theorems, including one that gives us necessary and sufficient conditions on the structure of quantum secret sharing schemes \cite{gottesman_theory_2000}.

Quantum error correction and quantum secret sharing go hand in hand. In essence, the goal of quantum error correction is to encode information in a way that allows the reconstruction of the original information in the absence of some part of the encoding. Quantum secret sharing has precisely the same goal with a few added constraints. And so, we are not so surprised that the main limitation of quantum secret sharing is a result of the \textit{no-cloning theorem}. The primary goal of this thesis is to explore ideas that can circumvent this limitation.

This is not a new idea. In 2004, Singh and Srikanth approached the problem by keeping a certain number of quantum shares with the dealer during the secret sharing procedure. They show that they are able to completely remove the restriction imposed by the no-cloning theorem \cite{singh_assisted_2004}. However, we have slight reservations with their approach, which are explained in \chref{ch:ss}. 

In this thesis, we present a different approach to loosening the restriction of the no-cloning theorem. We explore quantum secret sharing schemes that use more than one copy of a quantum state, and use a graph theoretical approach to characterize the class of schemes that become possible to implement. Specifically, we find that any quantum threshold scheme of the form $((t,2t-2+k,k))$ is valid under our construction; on the other hand, any scheme of the form $((t,2t-1+k,k))$ is not valid.

In Chapter 2, we present relevant background in quantum computation and quantum information. After that, we bring in some definitions and results from graph theory, which will be useful in our analysis. In Chapter 4, we present definitions and results related to classical and quantum secret sharing schemes. In Chapter 5, we introduce our approach to circumventing the no-cloning theorem using multiple copies of the quantum secret and show some preliminary results. In Chapter 6, we prove more general claims about our scheme, and present our main findings. Finally, we give a short corollary that provides a closed-form solution for the minimum number of shares that must reside with the dealer in Singh and Srikanth's \textit{Assisted Quantum Threshold Scheme}. In the final chapter, we discuss the implications of our findings and explore future work.