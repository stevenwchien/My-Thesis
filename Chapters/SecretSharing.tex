\chapter{Secret Sharing}
\label{ch:ss}

\section{Classical Secret Sharing}
\label{sec:css}

In this section we will describe secret sharing schemes and talk about some of its properties and definitions. A secret sharing scheme allows some \textit{secret} $\mathcal{S}$ to be divided into shares and distributed by some \textit{dealer} $\mathcal{D}$ to a set of \textit{participants} or \textit{players} $\mathcal{P}$. Each secret sharing scheme has a corresponding \textit{access structure} denoted by $\Gamma$. $\Gamma$ defines the set of \textit{authorized subsets} of $\mathcal{P}$ that have access to the secret. A subset of participants $A \in \Gamma$ should be able to reconstruct the original secret in its entirety, but a a set $B \notin \Gamma$ should have \textbf{no} information about the secret, in the sense that all possible values of the secret are equally likely. It is this final condition that makes implementing secret sharing schemes more difficult and interesting than just, say, taking a string and dividing it into equal chunks and giving one chunk to each person. Let us present some more formalized definitions:

\begin{definition}{Access Structure.}
    \label{defn:access-structure}
    An \textbf{access structure} is often denoted as $\Gamma$. An access structure specifies the set of all authorized subsets of players that are able to recover the secret.
\end{definition}

\begin{definition}{Authorized Set.}
    \label{defn:authorized-set}
    An \textbf{authorized set} is a subset $A \subseteq \mathcal{P}$ such that $A \in \Gamma$ for an access structure $\Gamma$.
\end{definition}

\begin{definition}{Monotone Access Structure.}
    \label{defn:monotone}
    An access structure $\Gamma$ is \textbf{monotone} if $B \in \Gamma$ and $B \subseteq C$ implies $C \in \Gamma$.
\end{definition}

\defref{defn:monotone} is particularly useful to us. Essentially, if some subset of participants satisfies the access structure, then all supersets of that subset should also satisfy the access structure. For the task of secret sharing, this definition is intuitive, if not necessary. Indeed, it would be difficult to imagine a procedure that implements an access structure that is \textit{not} monotone. Later on, we will see that the property of monotonicity is not only necessary for the realizability of an access structure, but it is also one of two sufficient conditions.

\begin{definition}{Minimal Access Structure.}
    \label{defn:minimal-as}
    An access structure $\Gamma$ is \textbf{minimal} if $A \in \Gamma$ implies for every $A' \in \Gamma \setminus \{A\}$, $A' \not\subset A$. Another name for this is an access structure's \textbf{basis}.
\end{definition}

\begin{definition}{Minimal Authorized Set.}
    \label{defn:minimal-authorized-set}
    A \textbf{minimal authorized set} is simply one of the authorized sets in the minimal access structure.
\end{definition}

Observe that any monotone access structure has a \textbf{unique minimal access structure}. If we assume that all of the access structures we are talking about are monotone, then this gives us a much simpler way to analyze access structures.

\begin{example}
    Let $\mathcal{P} = \{A,B,C,D\}$. Let $\Gamma_1 = \{(A,B), (A,C), (A,D)\}$. Let $\Gamma_2 = \{(A,B), (A,C), (A,D), (A,B,C)\}$. Assuming both access structures are monotone, then $\Gamma_1$ and $\Gamma_2$ represent the same access structure, because $(A,C) \subset (A,B,C)$. However, $\Gamma_1$ is a minimal access structure, and in fact, it is the \textbf{unique} minimal access structure of $\Gamma_2$. We say that $\Gamma_1$ and $\Gamma_2$ each has 3 minimal authorized sets.
\end{example}

\begin{definition}{Threshold Scheme.}
    \label{defn:threshold-scheme}
    A \textbf{$(t,n)$-threshold secret sharing scheme}, or just "threshold scheme" is a secret sharing scheme among $n$ players such that at least $t$ of those players must combine their respective shares in order to access the secret. The access structure for this scheme is composed of every subset of $\mathcal{P}$ of size $t$. This is also referred to as the set of all $t$-subsets of $n$.
\end{definition}

So, for our $(t,n)$ threshold scheme defined above, the \textit{access structure} can be defined more formally as such: $\Gamma = \{A | A \subseteq \mathcal{P} , |A| = t\}$.

\begin{example}
    Take the set of participants $\mathcal{P} = \{p_1,p_2,p_3\}$ might be $\Gamma = \{p_1p_2,p_2p_3,p_3p_1\}$. This access structure describes a $(2,3)$-threshold scheme.
\end{example}

As we mentioned above, Shamir developed one of the first implementations for a perfect threshold scheme based on polynomial interpolation \cite{shamir_how_1979}. A perfect secret sharing scheme is defined as one where authorized subsets have access to the secret, and unauthorized subsets have no information at all, in an information-theoretical sense, of the secret. Despite their shares, all possible values of the secret are possible.

Shamir's scheme is a $(t,n)$-threshold scheme. The scheme works as so. First, encode the secret as some number. Then, randomly generate a $(t-1)$-degree polynomial such that the constant term is number which encodes the secret. For each of the $n$ players, generate and distribute one of the pairs $i, p(i)$, where $i \in [1, \cdots, n]$. Each pair acts as a player's "share" of the secret. If $t$ or more players get together and pool together their shares, they can reconstruct the unique $(t-1)-$degree polynomial $p$ that generated those pairs, and then $p(0)$ reveals the secret. Note that having only $t-1$ or fewer pairs gives no information about the secret, because there would be infinitely many polynomials of degree $t-1$ passing through those $t-1$ or fewer pairs.