\chapter*{Introduction}
\label{introduction}

Secret sharing is a cryptographic procedure by which we take some sensitive data or a "secret" and distribute it among individual parties in such a way that access to the secret requires sufficiently large subsets of those parties to come together. Applications of these schemes are numerous, but some examples include giving a group of bank executives  access to a vault, requiring multiple officials to launch nuclear warheads, and creating secure voting procedures for a board of directors. In general, a secret sharing scheme is useful in a group of people where there is mutual distrust between its members, but they are forced to cooperate, either because there is information that is too sensitive for one person to access, or power must be distributed among its members in some way.

The moment that quantum computers become viable ways to store and perform operations on information, quantum computing algorithms and quantum cryptography will become extremely important. And so, thinking about quantum implementations of classical cryptographic procedures is a necessary and important exercise. Since 1999, researchers have been thinking about the implementation of secret sharing schemes in quantum computers. Hillery, Bu\u{z}ek, and Berthiaume presented one of the first schemes that involves using GHZ states to split quantum information into two parts such that both are necessary to recover the original information \cite{Hillery_1999}. Gottesman, Cleve, Lo use quantum error correcting codes to implement both threshold schemes and schemes with general access structures \cite{Cleve_1999}. Smith presents a construction for general access structures using monotone span programs. Later, Gottesman proves several important theorems regarding the realizability of quantum secret sharing schemes \cite{singh_assisted_2004}.

One of the main limitations of quantum secret sharing schemes come about as a consequence of the \textit{no-cloning theorem}. This theorem states that we cannot make a copy, or a clone, of an unknown quantum state. As we will later discuss, this restriction presents several practical limitations for applying secret sharing schemes to quantum computing. Therefore, the goal of this thesis is to explore ideas that will allow us to implement quantum threshold schemes that would otherwise be impossible to realize due to the no-cloning theorem. This idea is not new. Singh and Srikanth explore an idea that involves keeping a certain number of quantum shares with the dealer, and they show that they are able to completely remove the restriction imposed by the no-cloning theorem. 

In this paper, we present a new approach that achieves a similar goal of loosening the restriction of the no-cloning theorem. In Chapter 2, we will present relevant background for both classical and quantum secret sharing (QSS schemes). We will do things like prove the no-cloning theorem, and establish necessary quantum computing preliminaries. We will also bring in some other theory that will become useful in our analysis, such as some definitions and results from graph theory. In Chapter 3, we present our approach to circumventing the no-cloning theorem, and begin to reason about quantum threshold schemes (QTS) using multiple copies of the same state. In Chapter 4, we take our approach and prove some general claims about it. One of these claims characterizes the quantum threshold schemes that our new formulation can realize. Finally, we present a short corollary that provides a closed-form solution for the minimum number of shares that must reside with the dealer in Singh and Srikanth's \textit{Assisted Quantum Threshold Scheme} by using a well known result from graph theory. In the final chapter, we discuss the implications of our findings and explore possible future work.