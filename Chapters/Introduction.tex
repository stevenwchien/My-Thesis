\chapter*{Introduction}
\label{introduction}

Secret sharing is a cryptographic procedure by which we take some sensitive data and distribute it among participants in such a way that access to the secret requires sufficiently large subsets of those participants to come together. There are many situations in which these schemes are useful, and they are often characterized by: information that is too sensitive for just one person to have access; mutual distrust, yet a need to cooperate among members of a group; or the need to distribute power among multiple people. Some examples include giving a group of bank executives access to a vault, requiring multiple officials to launch nuclear warheads, and creating secure voting procedures for a board of directors. For this reason, developing secure and efficient methods of sharing secrets is an important task. 

The moment that quantum computers become viable ways to store and perform operations on information, quantum computing algorithms and quantum cryptography will become extremely important. There is concern that quantum computers will enable us to break certain encryption schemes, and to an extent, this is true. A popular example of a potentially insecure encryption scheme is RSA, whose security depends on the computational complexity of factoring large numbers. The concern is that there exists an efficient \textit{quantum} algorithm for factoring integers, rendering RSA insecure.

And so, quantum cryptography must develop with several goals in mind. The first is to provide quantum implementations of useful classical cryptographic procedures. This is important if we are to ever use quantum computers to manipulate, store, and transfer sensitive data. The second is to develop schemes that are secure not only against classical computers, but against quantum computers as well.

With this in mind, we turn our attention to secret sharing schemes on quantum information, also called \textit{quantum secret sharing} (QSS). Since 1999, researchers have thought about the implementation of secret sharing schemes on quantum information. Hillery, Bu\u{z}ek, and Berthiaume presented one of the first schemes that involves using GHZ states to split quantum information into two parts such that both are necessary to recover the original information \cite{Hillery_1999}. Gottesman, Cleve, Lo use quantum error correcting codes to implement both threshold schemes and schemes with general access structures \cite{Cleve_1999}. Smith presents a construction for general access structures using monotone span programs. Later, Gottesman proves several important theorems regarding the realizability of quantum secret sharing schemes \cite{singh_assisted_2004}.

One of the main limitations of quantum secret sharing schemes come about as a consequence of the \textit{no-cloning theorem}. This theorem states that we cannot make a copy, or a clone, of an unknown quantum state. As we will later discuss, this restriction presents several practical limitations for implementing secret sharing schemes in quantum computers. The primary goal of this thesis is to explore ideas that can circumvent the no-cloning theorem. This idea is not new. Singh and Srikanth explore an idea that involves keeping a certain number of quantum shares with the dealer during the procedure, and they show that they are able to completely remove the restriction imposed by the no-cloning theorem \cite{singh_assisted_2004}.

In this paper, we present a new approach that achieves a similar goal of loosening the restriction of the no-cloning theorem. In Chapter 1, we will present relevant background for both classical and quantum secret sharing (QSS schemes). We will do things like prove the no-cloning theorem, and establish necessary quantum computing preliminaries. We will also bring in some other theory that will become useful in our analysis, such as some definitions and results from graph theory. In Chapter 2, we present an approach to circumventing the no-cloning theorem that uses multiple copies of the quantum secret. Then, we present our results on the realizability of quantum threshold schemes. Finally, we present a short corollary providing a closed-form solution for the minimum number of shares that must reside with the dealer in Singh and Srikanth's \textit{Assisted Quantum Threshold Scheme} that follows from our more general results. In the final chapter, we discuss the implications of our findings and explore possible future work.