\chapter{Quantum Computation and Quantum Information}
\label{ch:qc-prelim}

\section{Quantum Computation}
\label{sec:qc}

An isolated \textbf{quantum system} is associated with a \textbf{Hilbert space} $\mathcal{H}$. The system is described completely by its \textbf{state vector}, which is a unit vector in its \textbf{state space} $\mathcal{H}$.

The simplest, but most applicable quantum mechanical system that we will use is the \textbf{qubit}, or the ``quantum bit''. The qubit lies in a two-dimensional Hilbert space, and is denoted as:

\begin{align}
    \ket{\psi} &= \alpha \ket{0} + \beta \ket{1}
\end{align}

where $|\alpha|^2 + |\beta|^2 = 1$. The normalization of the constant terms ensure that $\ket{\psi}$ is a unit vector. $\ket{0}$ and $\ket{1}$ form an orthonormal basis for the two-dimensional hilbert space. A general state vector in an $n$ dimensional Hilbert space is denoted:

\begin{align}
    \ket{\psi} &= \sum_{i=1}^n \alpha_i\ket{i}
\end{align}

The states $\ket{i}$ for $i \in \{1,...,n\}$ represent an orthonormal basis, and the constants $\alpha_1,...,\alpha_n$ are complex numbers normalized such that $\sum_{i=1}^n \alpha_i = 1$.

We say that the quantum state $\ket{\psi}$ is in a \textbf{superposition} of the states: $\{\ket{i}\}$.

The above formulation is called a state vector. Another formulation that represents a quantum state is a \textbf{density operator} or \textbf{density matrix}, generally denoted as $\rho$. One application where the density operator shines is in describing \textbf{mixed states}. A mixed state describes a quantum system in which there is uncertainty about its exact state. These are also known as ensembles of pure states: $\{\ket{\psi_i}, p_i\}$. $\ket{\psi_i}$ is one of the possible states the system is in, and the corresponding probability that the system is in state $\ket{\psi_i}$ is $p_i$. Then, the density operator describing this quantum subsystem is:

\begin{align}
    \rho &= \sum_{i} p_i * \ket{\psi_i}\bra{\psi_i}
\end{align}

Note that a mixed state is different than a superposition of states. A mixed state has to do with uncertainty about which state a quantum system is in. However, if a system is known to be in a superposition state, then there is no uncertainty. That state would be a pure state.

One of the most important applications of the density operator is the ability to describe quantum subsystems. Say that we have a density operator $\rho$ that represents the composite of two quantum systems $X$ and $Y$. Then, the density operator representing just the quantum system $X$ is:

\begin{align}
    \rho^X &\equiv \trace[Y]{\rho}
\end{align}

This is known as the \textbf(partial trace) over system $Y$, and the resulting $\rho^X$ is known as the \textbf{reduced density operator}. Note that if we perform a partial trace over subsystem $Y$, then we obtain the reduced density operator describing the subsystem $X$. The importance of the reduced density operator comes from the fact that it returns the correct measurements statistics for measurements done on subsystem $X$.

Another important technique that takes advantage of the density operator formulation is \textbf{purification}. This is a procedure in which we begin with some quantum system $A$ which is in a mixed state. We introduce a ``reference'' system $R$ such that the composite system $AR$ is in a pure state. Furthermore, the pure state $\ket{AR}$ reduces to $\rho^A$ when we trace over the reference system $R$.

A \textbf{quantum operation} is a unitary operator $U$ that acts on a quantum system denoted $\ket{\psi}$.

\begin{definition}{Unitary Operator.}
    A \textbf{unitary operator} $U;\mathcal{H} \to \mathcal{H}$ is a linear operator on a Hilbert space $\mathcal{H}$ that satisfies:
    
    \begin{align}
        U^*U = UU^* = I
    \end{align}
    
    Where $U^*$ is the adjoint of $U$.
\end{definition}

Examples of some commonly used operators are the 4 Pauli operators:

\begin{align}
    I &= \begin{pmatrix}
        1 & 0 \\ 
        0 & 1 \\ 
    \end{pmatrix} \\ 
    X &= \begin{pmatrix}
        0 & 1 \\ 
        1 & 0 \\ 
    \end{pmatrix} \\ 
    Y &= \begin{pmatrix}
        0 & i \\ 
        -i & 0 \\ 
    \end{pmatrix} \\ 
    Z &= \begin{pmatrix}
        1 & 0 \\ 
        0 & -1 \\ 
    \end{pmatrix}
\end{align}

If the evolution of a closed quantum system is described by the unitary operator $U$, and the system was originally in the state $\ket{\psi}$, then we denote the evolved system as:

\begin{align}
    \ket{\psi} \xrightarrow{U} U \ket{\psi}
\end{align}

In the density operator formulation, describing the ensemble of states where the closed quantum system was originally in state $\ket{\psi_i}$ with probability, $p_i$, the evolution is described as:

\begin{align}
    \rho = \sum_i p_i \ket{\psi_i}\bra{\psi_i} &\xrightarrow{U} \sum_i p_i U\ket{\psi_i}\bra{\psi_i}U^{\dagger} \\
    &\xrightarrow{U} U \left(\sum_i p_i \ket{\psi_i}\bra{\psi_i} \right)U^{\dagger} \\
    &\xrightarrow{U} U \rho U^{\dagger}
\end{align}

Any unitary matrix $U$ describes a valid quantum operation that can be used to evolve a quantum system. The unitary constraint comes from the requirement that quantum operations must be reversible.

\section{The No-Cloning Theorem}

One of the most important theorems that has a significant effect on quantum computing algorithms is the \textbf{no-cloning theorem}. Here, we present the no-cloning theorem with proof referencing Mermin's 2007 text \cite{mermin_quantum_2007}.

\begin{noclonetheorem}{}
    \label{thm:no-cloning-thm}
    Given an unknown, arbitrary quantum state $\psi$, there is no valid operator $U$ that can create an identical copy of this state. More formally there exists no operator such that $U(\ket{\psi} \ket{0}) = \ket{\psi}\ket{\psi}$.
\end{noclonetheorem}

\begin{proof}
    Assume for the sake of contradiction that there is such an operator. Then $U(\ket{\psi} \ket{0}) = \ket{\psi}\ket{\psi}$ and $U(\ket{\phi} \ket{0}) = \ket{\phi}\ket{\phi}$, for arbitrary quantum states $\ket{\psi}, \ket{\phi}$. Then:
    
    \begin{align}
        U(\alpha \ket{\psi} + \beta \ket{\phi})\otimes\ket{0} &= (\alpha \ket{\psi} + \beta \ket{\phi})\otimes (\alpha \ket{\psi} + \beta \ket{\phi}) \\ 
        &= \alpha^2\braket{\phi|\phi} + \beta^2\braket{\psi|\psi} + \alpha \beta \braket{\phi|\psi} + \alpha \beta \braket{\psi|\phi} \numberthis \label{eqn:no-clone-1}
    \end{align}
    
    But by linearity, we also have:
    
    \begin{align}
        U(\alpha \ket{\psi} + \beta \ket{\phi})\otimes\ket{0} &= \alpha U \ket{\psi}\ket{0} + \beta U \ket{\phi}\ket{0} \\ 
        &= \alpha \ket{\psi}\ket{\psi} + \beta \ket{\phi}\ket{\phi} \numberthis \label{eqn:no-clone-2}
    \end{align}
    
    \eqnref{eqn:no-clone-1} and \eqnref{eqn:no-clone-2} can only be the same if one of $\alpha$ or $\beta$ is equal to 0, which contradicts the assumption that $\ket{\psi}, \ket{\phi}$ are arbitrary.
\end{proof}

\begin{remark}
    Note that the theorem statement can also be made replacing $\ket{0}$ with an arbitrary state $\ket{e}$. What is important is that there is no unitary operator that acts as a ``general purpose copier''. For example, it would be easy to create an operator that can copy a state $\ket{\phi}$ if we know for a fact that the state is either $\ket{0}$ or $\ket{1}$ \cite{mermin_quantum_2007}. In general, a cloning device can only clone states which are orthogonal to each other \cite{nielsen_quantum_2010}.
\end{remark}

\section{Quantum Information Theory}
\label{section:qit}

When designing cryptographic schemes, it is important to work with quantitative definitions of information. This allows us to rigorously prove claims about the security of our procedures. An important concept in classical information theory is \textbf{entropy}. Entropy can be thought of as the amount of uncertainty you have about a quantity's value. In classical information theory, entropy is defined over probability distributions, which give you the probabilities that a quantity, like a random variable, takes on certain values.

This idea is transferable to quantum information, because quantum systems can be modelled as probability distributions, where the ``values'' that the system can take on are quantum states, and their associated probabilities are encoded in the density operator that represents the system.

Classical information theory uses \textbf{Shannon entropy}. The Shannon entropy of a probability distribution is defined as:

\begin{align}
    H(X) &\equiv \sum_x p_x \log (p_x)
\end{align}


The quantum analogue of Shannon entropy is called \textbf{von Neumann entropy}, and is defined below:

\begin{definition}{von Neumann Entropy.}
    The \textbf{von Neumann Entropy} of a quantum system $\rho$ is defined as:
    
    \begin{align}
        S(\rho) &\equiv -\trace{\rho \log(\rho)}
    \end{align}
    
    where $\trace{\rho}$ is the \textbf{trace} over $\rho$. The von Neumann entropy has an alternative definition using the eigenvalues of $\rho$:

    \begin{align}
        S(\rho) &= -\sum_x \lambda_x \log(\lambda_x)
    \end{align}
\end{definition}

\begin{remark}
    The von Neumann entropy $S(\rho)$ is non-negative.
\end{remark}

Because entropy is a measure of uncertainty, the von Neumann entropy of a pure state, where there is no uncertainty about the state of the system, is 0. On the other hand, the maximum entropy that a quantum system can have is when it is in a \textbf{maximally mixed state}, just as we would expect the Shannon entropy to be maximized over a uniform distribution. If a quantum system is in one of $d$ possible pure states, each with equal probability, then we say that it is in a maximally mixed state. The density operator for this system is:

\begin{align}
    \rho = \sum_{i=0}^{d-1} \frac{1}{d}\ket{i}\bra{i} = \frac{1}{d} I
\end{align}

And the von Neumann entropy of the maximally mixed state:

\begin{align}
    S(\rho) &= S(\tfrac{1}{d}I) = \log(d)
\end{align}

In quantum mechanics, we often work with two more quantum systems that interact with each other. An important entropy-like definition involving two or more systems is called the \textbf{relative entropy}, which measures the closeness of two probability distributions. In the case of quantum information, it measures the closeness of two quantum systems. We define quantum relative entropy below:

\begin{definition}{Quantum Relative Entropy.}
    Suppose you have two density operators $\rho$ and $\sigma$. The \textbf{relative entropy} of $\rho$ to $\sigma$ is:
    
    \begin{align}
        S(\rho||\sigma) &\equiv \trace{\rho \log(\rho)} - \trace{\rho \log (\sigma)}
    \end{align}
\end{definition}

\begin{remark}
    The relative von Neumann entropy of $\rho$ to $\sigma$ $S(\rho||\sigma)$ is non-negative.
\end{remark}

Another definition related to the interaction between systems is the \textbf{joint entropy}. Consider two quantum systems $X$ and $Y$. We denote the composite system of them to be $XY$, and their density operator to be $\rho^{XY}$.

\begin{definition}{Joint Entropy.}
    The \textbf{joint entropy} of a composite system $XY$ is simply the entropy defined on the density operator of the composite system:
    
    \begin{align}
        S(X,Y) = S(\rho^{XY}) &= -\trace{\rho^{XY} \log(\rho^{XY})}
    \end{align}
\end{definition}

As one might expect, the joint entropy can also be defined as function of the entropies of each system separately. To talk about that, we need to provide a couple more important definitions.

\begin{definition}{Conditional Entropy.}
    Say that we have two quantum systems $X,Y$, but we have full information about $Y$. Then we define the entropy of $X$ \textbf{conditional} on knowing $Y$ to be: 
    
    \begin{align}
        S(X|Y) &\equiv S(X,Y) - S(Y)
    \end{align}
\end{definition}

\begin{example}
    If we have a composite system $XY$ in a product state, that is, the state of the system can be written as $\rho \otimes \sigma$, then the joint entropy is:
\end{example}

\begin{align}
    S(X,Y) &= S(X) + S(Y)
\end{align}

This implies that the conditional entropy $S(X|Y) = S(X)$, which makes sense. If a composite system is in a product state, then there is no entanglement between the two systems. Information about one system does not give any information about the other.

Another quantity that relates to the information content of two systems is the \textbf{mutual information}. This is a measure of how much information the two systems have in common.

\begin{definition}{Mutual Information}
    The \textbf{mutual information} of two quantum systems $X,Y$ is defined as:
    
    \begin{align}
        I(X:Y) &\equiv S(X) + S(Y) - S(X,Y)
    \end{align}
\end{definition}

Using our definitions, one can see that the mutual information is also closely tied to the conditional entropy:

\begin{align}
        I(X:Y) &\equiv S(X) + S(Y) - S(X,Y) \\ 
        &= S(X) - S(X|Y) \\ 
        &= S(Y) - S(Y|X)
\end{align}
