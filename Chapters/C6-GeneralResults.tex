\chapter{General Results on Augmented Quantum Threshold Schemes}

\section{Generalizing the Union of Access Structures Approach}

In the previous chapter, we determined an upper bound on the value of $n$ with respect to $t$ at a fixed $k=2$. In this section, we ask the question of how $n$ varies with respect to $k$, and we prove more general results about the augmented quantum threshold scheme. 

At this point, there is hardly a pattern associated with the effect of $k$ on the possible values of $n$ and $t$, so we will have to take a different approach. What worked for us in the previous chapter was posing this problem as a graph coloring problem, and indeed, that is what we will do to start. Unfortunately, there does not exist a general theorem on the $c$-colorability of arbitrary graphs for $c > 3$ like there is for $c=2$. This problem is NP-hard. Nevertheless, it does not hurt to state clearly the following lemma:

\begin{lemma}
    \label{lem:k-color-access}
    The access structure graph complement corresponding to the augmented QTS $((t,n,k))$ is $k$-colorable if and only if $((t,n,k))$ is a valid augmented quantum threshold scheme.
\end{lemma}

This theorem and its proof are straightforward generalizations of those of \lemref{lem:2-color-access} from the previous section. In essence, the $k$-colorability of the graph ensures that we can separate it into at least $k$ stable sets, each of which can be assigned to one of the copies of the quantum state.

To keep our analysis going, let us consider an example with $k=3$ copies. The simplest, non-trivial example to consider is $((2,5,3))$. This is interesting to us because this specific $((t,n))$ pair is \textbf{not} possible with $k \leq 2$. We will show now that this scheme is indeed possible to construct. One way that we can implement it is to take our implementation of $((2,4,2))$ and add both an additional copy of the state and an additional player $p_5$. Then, we define the access structure for the newly added state, $\Gamma_3$, to contain all authorized subsets that include $p_5$. The access structures corresponding to the first two states stay unchanged from $((2,4,2))$. Observe that $\Gamma_3$ is a valid access structure because each authorized set contains the common player $p_5$. Since our original scheme is valid, then by construction, this new scheme is valid as well. We will now show that we can extend this construction indefinitely:

\begin{theorem}
    \label{thm:build-scheme} 
    Any scheme of the form $((t,2t-2+k,k))$ is a valid augmented quantum threshold scheme.
\end{theorem}

\begin{proof}
    We have shown in \thmref{thm:t-2t-2} that schemes of the form $((t, 2t, 2))$ are realizable. Assume, as an inductive hypothesis, that $((t, 2t - 2 + i, i))$ are valid augmented quantum threshold schemes. Now, let us consider the scheme $((t, 2t-1+i, i+1))$. Using our construction, we will have an access structure $\Gamma = \Gamma_1 \cup \Gamma_2 \cup ... \cup \Gamma_i \cup \Gamma_{i+1}$, where $\Gamma_r$ is the access structure corresponding to $\ket{\Psi}_r$ for $i \in {1,...,i}$. Let $\Gamma_1, ..., \Gamma_i$ remain unchanged from the scheme $((t, 2t - 2 + i, i))$. Then, we simply define $\Gamma_{i+1}$ to contain all of the authorized subsets that include the $2t-1+i$-th player. Then, this new access structure satisfies the no-cloning theorem, and by the IH, all of the access structures satisfy the no-cloning theorem. So $((t, 2t-1+i, i+1))$ is a valid augmented quantum threshold scheme.
\end{proof}

\subsection{Can we do better?}
\label{ssec:better}

\thmref{thm:build-scheme} gives us a more general result, but the question still remains - can we do better? Unfortunately, the answer is in the negative. To show this, we will bring in the Kneser Graph, introduced in \defref{defn:kneser-graph}.

\begin{remark}
    Observe that the AS graph complement of a QTS of the form $((t,n))$ is a $K(n,t)$. It is a Kneser graph on $n$ elements with subsets of size $t$. This is independent of the number of copies $k$.
\end{remark}

\corref{cor:no} connects Kneser's conjecture and the chromatic number of Kneser graphs. We can use this to prove the following theorem:

\begin{theorem}
    \label{thm:no-more} 
    Any scheme of the form $((t,2t-1+k,k))$ is not a valid augmented quantum threshold scheme.
\end{theorem}

\begin{proof}
    The complement of the AS graph for the access structure corresponding to the threshold scheme $((t,2t-1+k,k))$ is a $K(2t-1+k,t)$, or a Kneser Graph on a set of $2t-1+k$ elements, with subsets of size $t$. Then by \corref{cor:no}, the corresponding graph is not $k$-colorable. By \thmref{lem:k-color-access}, schemes of the form $((t,2t-1+k,k))$ are not valid augmented quantum threshold schemes.
\end{proof}

So, this is the best that we can do with $k$ copies of a quantum state, using our union of access structures approach. More specifically, we have shown that for every extra copy of the quantum state that we begin with, we are only able to implement one additional threshold scheme for a fixed group size $n$. The good news is that we have a closed form answer for the question: how many copies of the quantum state do we need to implement an augmented quantum threshold scheme $((t,n,k))$? For $n \geq 2t$, we need $k = n - 2t + 2$.

In a way, this result might be expected with our particular construction. The number of authorized sets in the access structure increases exponentially for every new person added, so it makes sense that the access structures that we can implement are constrained even when we add more copies.

\section{Security Considerations}

An analysis of a cryptographic scheme would not be complete without an analysis of its security. Because we construct our scheme using a union of access structures, all we need to do is show that an augmented QTS is no less secure than its constituent parts. That is, using our union of access structures strategy, if the QSS schemes that are used with each copy of the quantum state are perfect, then the resulting scheme is perfect as well.

\begin{lemma}
    \label{lem:t-2t-1}
    A quantum threshold scheme of the form $((t,2t-1))$ is a perfect quantum threshold scheme.
\end{lemma}

\begin{proof}
    Observe that this is a maximal access structure. Any authorized set must have at least $t$ participants. Therefore, the complement of any authorized set must have a size of $t-1$ or smaller. Which means that the authorized sets are exactly the complements of the unauthorized sets, so this scheme is perfect by \cite{gottesman_theory_2000}.
\end{proof}

\begin{theorem}
    An augmented QTS of the form $((t,2t-2+k,k))$ is a perfect quantum threshold scheme.
\end{theorem}

\begin{proof}
    We are going to prove this using induction. We will show that the scheme is perfect by showing that the underlying schemes that compose it are perfect. For our base case, consider the QTS $((t,2t-1))$. Note that there is an implicit $k=1$. Then by \lemref{lem:t-2t-1}, $((t,2t-1))$ is a perfect scheme.
    
    For our inductive hypothesis, assume that $((t,2t-2+k,k))$ is a perfect QTS. We will show that $((t,2t-1+k,k+1))$ is also a perfect QTS. Using our construction, the $((t,2t-1+k,k+1))$ is built by taking the original scheme $((t,2t-2+k,k))$, and then adding one more share $\ket{\psi_{k+1}}$ and one more player $P_{2t-1+k}$. $((t,2t-2+k,k))$ is a perfect secret sharing scheme by the IH. Therefore, we only need to show that the access structure corresponding to $\ket{\psi_{k+1}}$ is perfect. We will call this access structure $\Gamma_{k+1}$. Observe that this access structure contains all of the $t$-subsets of $2t-1+k$ that contain $P_{2t-1+k}$. However, this access structure is not maximal. Namely, the subset of players $\mathcal{P} \setminus P_{2t-1+k} = \{P_1,P_2,...,P_{2t-2+k}\}$ is unauthorized, as well as its complement $\{P_{2t-1+k}\}$. However, we can construct this access structure out of maximal access structures using a construction presented by Gottesman in \cite{gottesman_theory_2000}.
    
    First, let the maximal access structure containing $\Gamma_{k+1}$ be $\Gamma^M$. Then, $\Gamma_M = \Gamma_{k+1} \cup \{P_1,P_2,...,P_{2t-2+k}\}$. This is a maximal access structure because the authorized sets are exactly the complements of the unauthorized sets. Let's denote the number of minimal authorized sets in $\Gamma_{k+1}$ to be $r$, where $r = \binom{t}{2t-1+k} + 1$. Then, we construct a layered scheme, using an $((r,2r-1))$ scheme as the outermost layer. Let the shares of this scheme be $S_i$ for $i \in \{1,...,2r-1\}$. In the first $r$ shares, there is one share corresponding to each minimal authorized set. For each of these first $r$ shares: $\{S_j\}_{j\in 1,...,r}$, we are going to use a secret sharing scheme on $S_j$ to share among the participants in each authorized set $A_j$. Each scheme will be a $((l_j,l_j))$ threshold scheme, where $l_j = |A_j|$ is the number of players in the corresponding authorized set. The latter $r-1$ shares each have an access structure equal to $\Gamma^M$. In this way, any authorized set in $\Gamma$, $A_1,\dots,A_r$, will be able to recover one of the first $r$ shares, and will also be able to recover the latter $r-1$ shares because $\Gamma \subseteq \Gamma^M$. This gives any authorized set in $\Gamma$ enough shares to recover the secret. However, an authorized set that is in $\Gamma^M$ but not in $\Gamma$ can only recover $r-1$ shares, which is not enough to recover the original secret.
    
    Note that any scheme of the form $((r,2r-1))$ is maximal, as is any scheme of the form $((t,t))$. The authorized sets are the complements of the unauthorized sets. So, we have found a way to construct $\Gamma_{k+1}$ using only maximal access structures. Therefore, $\Gamma_{k+1}$ corresponds to a perfect secret sharing scheme, so by induction $((t,2t-2+k,k))$ is a perfect quantum threshold scheme.
\end{proof}

\section{Back to Assisted Quantum Secret Sharing}
\label{sec:aqss-and-aqss}

In this section, we give a short result that provides a closed form solution for the number of resident shares needed in an assisted QSS implementation of a threshold scheme. Recall the assisted quantum secret sharing scheme presented by Singh and Srikanth \cite{singh_assisted_2004}. The number of resident shares needed in their scheme is $\lambda-1$, where $\lambda$ is the minimum number of partially linked classes there are in the AS graph of the access structure $\Gamma$. The question that is left open is finding a way to compute $\lambda$. We show that our results from the previous section can provide a closed form answer to this question when the access structure corresponds to a threshold scheme.

Let us consider a quantum threshold scheme $((t,n))$. How many resident shares are needed to realize this scheme? By \propref{prop:chrom-clique}, the minimum number of partially linked classes in the AS graph is the same as the chromatic number of the AS graph's complement. So, by Kneser's Conjecture, finding this for quantum threshold schemes is easy. The AS graph complement of this scheme is a Kneser graph on $n$ elements with subsets of size $t$. The chromatic number of the AS graph complement is $n-2t+2$, so $\lambda = n-2t+2$. Therefore, the minimum number of resident shares we need for an assisted quantum threshold scheme on $n$ players with threshold $t$ is $n-2t+1$.